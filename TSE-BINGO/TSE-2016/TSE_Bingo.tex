
%% bare_jrnl_compsoc.tex
%% V1.4b
%% 2015/08/26
%% by Michael Shell
%% See:
%% http://www.michaelshell.org/
%% for current contact information.
%%
%% This is a skeleton file demonstrating the use of IEEEtran.cls
%% (requires IEEEtran.cls version 1.8b or later) with an IEEE
%% Computer Society journal paper.
%%
%% Support sites:
%% http://www.michaelshell.org/tex/ieeetran/
%% http://www.ctan.org/pkg/ieeetran
%% and
%% http://www.ieee.org/

%%*************************************************************************
%% Legal Notice:
%% This code is offered as-is without any warranty either expressed or
%% implied; without even the implied warranty of MERCHANTABILITY or
%% FITNESS FOR A PARTICULAR PURPOSE!
%% User assumes all risk.
%% In no event shall the IEEE or any contributor to this code be liable for
%% any damages or losses, including, but not limited to, incidental,
%% consequential, or any other damages, resulting from the use or misuse
%% of any information contained here.
%%
%% All comments are the opinions of their respective authors and are not
%% necessarily endorsed by the IEEE.
%%
%% This work is distributed under the LaTeX Project Public License (LPPL)
%% ( http://www.latex-project.org/ ) version 1.3, and may be freely used,
%% distributed and modified. A copy of the LPPL, version 1.3, is included
%% in the base LaTeX documentation of all distributions of LaTeX released
%% 2003/12/01 or later.
%% Retain all contribution notices and credits.
%% ** Modified files should be clearly indicated as such, including  **
%% ** renaming them and changing author support contact information. **
%%*************************************************************************


% *** Authors should verify (and, if needed, correct) their LaTeX system  ***
% *** with the testflow diagnostic prior to trusting their LaTeX platform ***
% *** with production work. The IEEE's font choices and paper sizes can   ***
% *** trigger bugs that do not appear when using other class files.       ***                          ***
% The testflow support page is at:
% http://www.michaelshell.org/tex/testflow/


\documentclass[10pt,journal,compsoc]{IEEEtran}
%
% If IEEEtran.cls has not been installed into the LaTeX system files,
% manually specify the path to it like:
% \documentclass[10pt,journal,compsoc]{../sty/IEEEtran}



\usepackage{booktabs}
\usepackage[table,xcdraw]{xcolor}


\usepackage{etex}
\usepackage{picture}
\usepackage{multirow}
\usepackage{seqsplit}
\usepackage{color}
\usepackage{listings}
\usepackage{graphicx}
\usepackage{caption}
\usepackage{microtype}
\usepackage[ruled,vlined,linesnumbered]{algorithm2e}
\newcommand\mycommfont[1]{\fontsize{7}{7}\selectfont\rmfamily\textcolor{gray}{#1}}
\SetCommentSty{mycommfont}
\usepackage{setspace}
\usepackage{breakurl}
\usepackage{amssymb}
\usepackage{amsmath}
\usepackage{pifont}
\usepackage{hyphenat}
\usepackage{xspace}
\usepackage{setspace}
\let\labelindent\relax
\usepackage{enumitem}
\usepackage{times}
\usepackage{soul}
\usepackage{color}
\usepackage{balance}
\usepackage{tikz}
\usepackage{threeparttable}
\usepackage{mathtools}
\usepackage{tabularx}
\usepackage{scrextend}
\usepackage{varwidth}
\usetikzlibrary{calc,decorations.pathmorphing,shapes}
\usepackage[colorlinks=true,citecolor=blue,linkcolor=blue,urlcolor=blue]{hyperref}

\usepackage{verbatim}
\usepackage{caption}
\usetikzlibrary{arrows}
\newcommand{\powerset}[1]{\mathbb{P}(#1)}
\usepackage{subfigure}
\usepackage{listings}

\lstdefinelanguage
   [x64]{Assembler}     % add a "x64" dialect of Assembler
   [x86masm]{Assembler} % based on the "x86masm" dialect
   % with these extra keywords:
   {morekeywords={CDQE,CQO,CMPSQ,CMPXCHG16B,JRCXZ,LODSQ,MOVSXD, %
                  POPFQ,PUSHFQ,SCASQ,STOSQ,IRETQ,RDTSCP,SWAPGS, %
                  rax,rdx,rcx,rbx,rsi,rdi,rsp,rbp, %
                  r8,r8d,r8w,r8b,r9,r9d,r9w,r9b}} % etc.

\lstset{language=[x64]Assembler}


\ifdefined \dependency
	\newcommand{\idiomDep}[1]{\textcolor{orange}{#1}}

\else
	\newcommand{\idiomDep}[1]{}
\fi

\setlength{\textfloatsep}{0pt}

\newtheorem{example}{Example}

\newlength\marincrease

\makeatletter
\newenvironment{MyAlgo}[2][htbp]
  {\renewcommand{\@algocf@start}{%
    \setlength\marincrease{#2}
  \@algoskip%
  \begin{lrbox}{\algocf@algobox}%
  \begin{minipage}{\dimexpr\textwidth+2\marincrease\relax}
  \setlength{\algowidth}{\hsize}%
  \vbox\bgroup% save all the algo in a box
  \hbox to\algowidth\bgroup\hbox to \algomargin{\hfill}\vtop\bgroup%
  \ifthenelse{\boolean{algocf@slide}}{\parskip 0.5ex\color{black}}{}%
  % initialization
  \addtolength{\hsize}{-1.5\algomargin}%
  \let\@mathsemicolon=\;\def\;{\ifmmode\@mathsemicolon\else\@endalgoln\fi}%
  \raggedright\AlFnt{}%
  \ifthenelse{\boolean{algocf@slide}}{\IncMargin{\skipalgocfslide}}{}%
  \@algoinsideskip%
%   \let\@emathdisplay=\]\def\]{\algocf@endline\@emathdisplay\nl}%
  }%
\renewcommand{\@algocf@finish}{%
  \@algoinsideskip%
  \egroup%end of vtop which contain all the text
  \hfill\egroup%end of hbox wich contains [margin][vtop]
  \ifthenelse{\boolean{algocf@slide}}{\DecMargin{\skipalgocfslide}}{}%
  %
  \egroup%end of main vbox
  \end{minipage}
  \end{lrbox}%
  \makebox[\linewidth][c]{\algocf@makethealgo}% print the algo
  \@algoskip%
  % restore dimension and macros
  \setlength{\hsize}{\algowidth}%
  \lineskip\normallineskip\setlength{\skiptotal}{\@defaultskiptotal}%
  \let\;=\@mathsemicolon%
  \let\]=\@emathdisplay%
}%
  \begin{algorithm}[#1]}
  {\end{algorithm}}
\makeatother

%\usepackage{subcaption}

\newcommand{\tool}{\textsc{BinGo}\xspace}
\newcommand{\toolNew}{\textsc{BinGo-E}\xspace}
\makeatletter
\def\@begintheorem#1#2{\trivlist
  \item[\hskip \labelsep{ #1\ #2.}]}%\bfseries
\makeatother

\newtheorem{defn}{Definition}
\newtheorem{assu}{Assumption}
\newtheorem{exmp}{Example}
\newtheorem{pattern}{Rule}

\clubpenalty = 10000
\widowpenalty = 10000
\displaywidowpenalty = 10000



\begin{document}
%
% paper title
% Titles are generally capitalized except for words such as a, an, and, as,
% at, but, by, for, in, nor, of, on, or, the, to and up, which are usually
% not capitalized unless they are the first or last word of the title.
% Linebreaks \\ can be used within to get better formatting as desired.
% Do not put math or special symbols in the title.
\title{Accurate and Scalable Cross-Architecture Cross-OS Binary Code Search with Emulation}
%
%
% author names and IEEE memberships
% note positions of commas and nonbreaking spaces ( ~ ) LaTeX will not break
% a structure at a ~ so this keeps an author's name from being broken across
% two lines.
% use \thanks{} to gain access to the first footnote area
% a separate \thanks must be used for each paragraph as LaTeX2e's \thanks
% was not built to handle multiple paragraphs
%
%
%\IEEEcompsocitemizethanks is a special \thanks that produces the bulleted
% lists the Computer Society journals use for "first footnote" author
% affiliations. Use \IEEEcompsocthanksitem which works much like \item
% for each affiliation group. When not in compsoc mode,
% \IEEEcompsocitemizethanks becomes like \thanks and
% \IEEEcompsocthanksitem becomes a line break with idention. This
% facilitates dual compilation, although admittedly the differences in the
% desired content of \author between the different types of papers makes a
% one-size-fits-all approach a daunting prospect. For instance, compsoc
% journal papers have the author affiliations above the "Manuscript
% received ..."  text while in non-compsoc journals this is reversed. Sigh.

\author{Yinxing Xue,~\IEEEmembership{Non-member,}
       Zhengzi Xu,~\IEEEmembership{Non-member,}
       Mahinthan Chandramohan,~\IEEEmembership{Non-member,}
       Yang Liu,~\IEEEmembership{Member,~IEEE,}
       and Chia Yuan Cho,~\IEEEmembership{Non-member,}
       %  and~Tan Hee Beng Kuan,~\IEEEmembership{Member,~IEEE}% <-this % stops a space
%\IEEEcompsocitemizethanks{\IEEEcompsocthanksitem This research is supported in part by the National Research Foundation, Singapore under its National Cybersecurity R\&D Program (Award No. NRF2014NCR-NCR001-30).}
 \IEEEcompsocitemizethanks{
 \IEEEcompsocthanksitem This research is supported in part by the National Research Foundation, Singapore under its National Cybersecurity R\&D Program (Award No. NRF2014NCR-NCR001-30).
 \IEEEcompsocthanksitem  Y. Xue,  Z. Xu,  M. Chandramohanare and Y. Liu are with Nanyang Technological University, Singapore, 637553. E-mail:{tslxuey@ntu.edu.sg, XU0001ZI@e.ntu.edu.sg, mahintha001@e.ntu.edu.sg, yangliu@ntu.edu.sg}.
 %\IEEEcompsocthanksitem are with School of Computer Science and Engineering, Nanyang Technological University, Singapore, 639798. E-mail:{}
 \IEEEcompsocthanksitem C.Y. Cho is with DSO National Laboratories, Singapore, 118230. E-mail:{cchiayua@dso.org.sg} }
\thanks{Manuscript received April XX, 2017; revised XXXX XX, 2017.}}

% note the % following the last \IEEEmembership and also \thanks -
% these prevent an unwanted space from occurring between the last author name
% and the end of the author line. i.e., if you had this:
%
% \author{....lastname \thanks{...} \thanks{...} }
%                     ^------------^------------^----Do not want these spaces!
%
% a space would be appended to the last name and could cause every name on that
% line to be shifted left slightly. This is one of those "LaTeX things". For
% instance, "\textbf{A} \textbf{B}" will typeset as "A B" not "AB". To get
% "AB" then you have to do: "\textbf{A}\textbf{B}"
% \thanks is no different in this regard, so shield the last } of each \thanks
% that ends a line with a % and do not let a space in before the next \thanks.
% Spaces after \IEEEmembership other than the last one are OK (and needed) as
% you are supposed to have spaces between the names. For what it is worth,
% this is a minor point as most people would not even notice if the said evil
% space somehow managed to creep in.

% The paper headers
\markboth{IEEE Transaction on Software Engineering,~Vol.~XX, No.~X, April~2017}%
{Shell \MakeLowercase{\textit{et al.}}: Bare Demo of IEEEtran.cls for Computer Society Journals}
% The only time the second header will appear is for the odd numbered pages
% after the title page when using the twoside option.
%
% *** Note that you probably will NOT want to include the author's ***
% *** name in the headers of peer review papers.                   ***
% You can use \ifCLASSOPTIONpeerreview for conditional compilation here if
% you desire.



% The publisher's ID mark at the bottom of the page is less important with
% Computer Society journal papers as those publications place the marks
% outside of the main text columns and, therefore, unlike regular IEEE
% journals, the available text space is not reduced by their presence.
% If you want to put a publisher's ID mark on the page you can do it like
% this:
%\IEEEpubid{0000--0000/00\$00.00~\copyright~2015 IEEE}
% or like this to get the Computer Society new two part style.
%\IEEEpubid{\makebox[\columnwidth]{\hfill 0000--0000/00/\$00.00~\copyright~2015 IEEE}%
%\hspace{\columnsep}\makebox[\columnwidth]{Published by the IEEE Computer Society\hfill}}
% Remember, if you use this you must call \IEEEpubidadjcol in the second
% column for its text to clear the IEEEpubid mark (Computer Society jorunal
% papers don't need this extra clearance.)



% use for special paper notices
%\IEEEspecialpapernotice{(Invited Paper)}



% for Computer Society papers, we must declare the abstract and index terms
% PRIOR to the title within the \IEEEtitleabstractindextext IEEEtran
% command as these need to go into the title area created by \maketitle.
% As a general rule, do not put math, special symbols or citations
% in the abstract or keywords.

%\renewcommand{\algorithmicrequire}{ \textbf{Input:}} %Use Input in the format of Algorithm
%\renewcommand{\algorithmicensure}{ \textbf{Output:}} %UseOutput in the format of Algorithm
%\newcommand{\beh}{behavior\xspace}
%\newcommand{\Beh}{Behavior\xspace}
%\newcommand{\behs}{behaviors\xspace}
%\newcommand{\Behs}{Actions\xspace}
%\newcommand{\aly}{analyze\xspace}
%\newcommand{\alys}{analyzes\xspace}
%\newcommand{\tool}{\textsc{AttackClone}\xspace}
%\newcommand{\graph}{Attack Graph\xspace}
%\def\WithComments{}
\ifdefined \WithComments
	\newcommand{\xyx}[1]{\textcolor{red}{#1}}
    \newcommand{\wjj}[1]{\textcolor{blue}{\textbf{ToChange: #1}}}
    \newcommand{\ly}[1]{\textcolor{darkred}{\textbf{#1}}}
\else
	\newcommand{\xyx}[1]{\textcolor{black}{#1}}
    \newcommand{\wjj}[1]{}
    \newcommand{\ly}[1]{}
\fi
\newcommand{\todo}[1]{\textbf{\textcolor{cyan}{TODO: #1}}}
\newcommand{\note}[1]{\textcolor{red}{#1}}
\newcommand{\mahin}[1]{\textcolor{blue}{#1}}
%\newcommand{\comment}[1]{\textbf{\textcolor{green}{Comment: #1}}}
\newcommand{\alarm}[1]{\textbf{\textcolor{red}{Alarm: #1}}}
\newcommand{\deprecate}[1]{\textbf{\textcolor{red}{\{Deprecated: #1\}}}}
\newcommand{\question}[1]{\textbf{\textcolor{blue}{Question: #1}}}
\newcommand{\remove}[1]{}
\newcommand{\leftmapsto}{\leftarrow\!\shortmid}
\newcommand{\pluseq}{\mathrel{+}=}

%\st{overstriking}
\newcommand*\widefbox[1]{\fbox{\hspace{2em}#1\hspace{2em}}}


%\setitemize[0]{leftmargin=10pt}
\newcommand{\fonttt}[1]{\begin{ttfamily}#1\end{ttfamily}}
\setitemize[0]{leftmargin=10pt}

\newtheorem{mydef}{\textbf{Definition}}%[section]
\definecolor{pblue}{rgb}{0.13,0.13,1}
\definecolor{pgreen}{rgb}{0,0.5,0}
\definecolor{pred}{rgb}{0.9,0,0}
\definecolor{pgray}{rgb}{0.46,0.45,0.48}
\definecolor{ppurple}{rgb}{1,0.2,1}
\definecolor{pblack}{rgb}{0,0,0}
\lstset{
	basicstyle=\scriptsize\tt,
	tabsize=4,
	showstringspaces=false,
	columns=flexible,
	commentstyle=\color{pgreen},
  	keywordstyle=\color{pblue},
  	stringstyle=\color{ppurple},
	breaklines=true,
	language=Java,
    showspaces=false,
    numbers=left,                    % where to put the line-numbers; possible values are (none, left, right)
    numbersep=5pt,                   % how far the line-numbers are from the code
    numberstyle=\tiny\color{pblack} % the style that is used for the line-numbers
}

\makeatletter
\newcommand*{\da@rightarrow}{\mathchar"0\hexnumber@\symAMSa 4B }
\newcommand*{\da@leftarrow}{\mathchar"0\hexnumber@\symAMSa 4C }
\newcommand*{\xdashrightarrow}[2][]{%
  \mathrel{%
    \mathpalette{\da@xarrow{#1}{#2}{}\da@rightarrow{\,}{}}{}%
  }%
}
\newcommand{\xdashleftarrow}[2][]{%
  \mathrel{%
    \mathpalette{\da@xarrow{#1}{#2}\da@leftarrow{}{}{\,}}{}%
  }%
}
\newcommand*{\da@xarrow}[7]{%
  % #1: below
  % #2: above
  % #3: arrow left
  % #4: arrow right
  % #5: space left
  % #6: space right
  % #7: math style
  \sbox0{$\ifx#7\scriptstyle\scriptscriptstyle\else\scriptstyle\fi#5#1#6\m@th$}%
  \sbox2{$\ifx#7\scriptstyle\scriptscriptstyle\else\scriptstyle\fi#5#2#6\m@th$}%
  \sbox4{$#7\dabar@\m@th$}%
  \dimen@=\wd0 %
  \ifdim\wd2 >\dimen@
    \dimen@=\wd2 %
  \fi
  \count@=2 %
  \def\da@bars{\dabar@\dabar@}%
  \@whiledim\count@\wd4<\dimen@\do{%
    \advance\count@\@ne
    \expandafter\def\expandafter\da@bars\expandafter{%
      \da@bars
      \dabar@
    }%
  }%
  \mathrel{#3}%
  \mathrel{%
    \mathop{\da@bars}\limits
    \ifx\\#1\\%
    \else
      _{\copy0}%
    \fi
    \ifx\\#2\\%
    \else
      ^{\copy2}%
    \fi
  }%
  \mathrel{#4}%
}
\makeatother


%%%%%%%%%%%%%%%%%%COMMANDS TO REDUCE THE SPACE%%%%%%%%%%%%%%%%%%%%%
%\addtolength{\parskip}{-1mm}
%\addtolength{\floatsep}{-6mm}
%\addtolength{\textfloatsep}{-6mm}
%\addtolength{\abovecaptionskip}{-0.5mm}
%\addtolength{\belowcaptionskip}{-0.5mm}
%Source: http://tex.stackexchange.com/questions/60216/how-to-create-a-squiggle-arrow-with-some-text-on-it-in-tikz
\newcounter{sarrow}
\newcommand\xrsquigarrow[1]{%
\stepcounter{sarrow}%
\mathrel{\begin{tikzpicture}[baseline= {( $ (current bounding box.south) + (0,-0.5ex) $ )}]
\node[inner sep=.5ex] (\thesarrow) {$\scriptstyle #1$};
\path[draw,<-,decorate,
  decoration={zigzag,amplitude=0.7pt,segment length=1.2mm,pre=lineto,pre length=4pt}]
    (\thesarrow.south east) -- (\thesarrow.south west);
\end{tikzpicture}}%
}


\IEEEtitleabstractindextext{%
\begin{abstract}
Different from source code clone detection, clone detection (similar code search) in binary executables faces big challenges due to the gigantic differences in the syntax and the structure of  binary code that result  from different configurations of compilers, architectures and OSs. Existing studies have proposed different categories of features for detecting binary code clones, including CFG structures, n-gram in CFG, input/output values, etc. In our previous study and the tool \tool, to mitigate the huge gaps in CFG structures due to different compilation scenarios, we propose a selective inlining technique to capture the complete function semantics by inlining relevant library and user-defined functions. However, only features of input/output values are considered in \tool. In this study, we propose to incorporate features from different categories (e.g., structural features and high-level semantic features) for accuracy improvement and emulation for efficiency improvement. We empirically compare our tool, \toolNew, with the pervious tool \tool~and the available state-of-the-art tools of binary code search in terms of search accuracy and performance. \xyx{Results show that \toolNew achieves significantly better accuracies than \tool for cross-architecture matching, cross-OS matching, cross-compiler matching and intra-compiler matching. Additionally, in the new task of matching binaries of forked projects, \toolNew also exhibits a better accuracy than the existing benchmark tool. Meanwhile, \toolNew takes less time than \tool during the process of matching.}
%However, developing an accurate and effective binary code search tool is challenging due to the gigantic syntax and structural differences in binaries resulted from different compilers, architectures and OSs.
%In this paper, we propose \tool --- a scalable and robust binary search engine supporting various architectures and OSs.
%The key contribution is a selective inlining technique to capture the complete function semantics by inlining relevant library and user-defined functions.
%In addition, architecture and OS neutral function filtering is proposed to dramatically reduce the irrelevant target functions.
%Besides, we introduce length variant partial traces to model binary functions in a program structure agnostic fashion.
%The experimental results show that \tool can find semantic similar functions across architecture and OS boundaries, even with the presence of program structure distortion, in a scalable manner. Using \tool, we also discovered a zero-day vulnerability in Adobe PDF Reader, a COTS binary.

\end{abstract}

% Note that keywords are not normally used for peerreview papers.
\begin{IEEEkeywords}
Binary Code Search, Binary Clone Detection, Vulnerability Matching, Emulation, 3D-CFG
\end{IEEEkeywords}}


% make the title area
\maketitle


% To allow for easy dual compilation without having to reenter the
% abstract/keywords data, the \IEEEtitleabstractindextext text will
% not be used in maketitle, but will appear (i.e., to be "transported")
% here as \IEEEdisplaynontitleabstractindextext when the compsoc
% or transmag modes are not selected <OR> if conference mode is selected
% - because all conference papers position the abstract like regular
% papers do.
\IEEEdisplaynontitleabstractindextext
% \IEEEdisplaynontitleabstractindextext has no effect when using
% compsoc or transmag under a non-conference mode.



% For peer review papers, you can put extra information on the cover
% page as needed:
% \ifCLASSOPTIONpeerreview
% \begin{center} \bfseries EDICS Category: 3-BBND \end{center}
% \fi
%
% For peerreview papers, this IEEEtran command inserts a page break and
% creates the second title. It will be ignored for other modes.
\IEEEpeerreviewmaketitle



%\IEEEraisesectionheading{\section{Introduction}\label{sec:introduction}}
% Computer Society journal (but not conference!) papers do something unusual
% with the very first section heading (almost always called "Introduction").
% They place it ABOVE the main text! IEEEtran.cls does not automatically do
% this for you, but you can achieve this effect with the provided
% \IEEEraisesectionheading{} command. Note the need to keep any \label that
% is to refer to the section immediately after \section in the above as
% \IEEEraisesectionheading puts \section within a raised box.




% The very first letter is a 2 line initial drop letter followed
% by the rest of the first word in caps (small caps for compsoc).
%
% form to use if the first word consists of a single letter:
% \IEEEPARstart{A}{demo} file is ....
%
% form to use if you need the single drop letter followed by
% normal text (unknown if ever used by the IEEE):
% \IEEEPARstart{A}{}demo file is ....
%
% Some journals put the first two words in caps:
% \IEEEPARstart{T}{his demo} file is ....
%
% Here we have the typical use of a "T" for an initial drop letter
% and "HIS" in caps to complete the first word.
%\IEEEPARstart{T}{his} demo file is intended to serve as a ``starter file''
%for IEEE Computer Society journal papers produced under \LaTeX\ using
%IEEEtran.cls version 1.8b and later.
% You must have at least 2 lines in the paragraph with the drop letter
% (should never be an issue)
%\hfill mds
%\hfill August 26, 2015


%aum ganathipathaye namaha
%sri rama jeyam
\section{Introduction}\label{sec:info}

%\ly{we should write it more general first for SE communities.}

\IEEEPARstart{W}{ith} the prosperity of open source projects and open software under GPL, new software (even commercial software) can reuse the code of the existing related projects. However, extensive reuse of existing code leads to some legal and security issues. According to the report on \url{`gpl-violations.org'} \cite{gpl_vio}, over 200 cases of GNU public license violations  have been identified in software products. Among them, \textsc{VMware} is the well-known software vendor that is facing the lawsuit regarding the GPL violation \cite{vm_news}. Meanwhile, code reuse brings risks that a vulnerability disclosed in reused projects can spread into the
new commercial products as undisclosed vulnerability. For example, in Dec. 2014,  vulnerabilities in the implementation of the network time protocol (NTP) \emph{ntpd} affected the products that import it, such as Linux distribution, Cisco's 5900x and Apple's OSX switches. %In addition, cyber attackers may reuse the existing malicious code in creating malware.

%\mahin{GPL license violation is a good application of this work. at the moment, we don't have any experiment result to substantiate that. Do you know any known cases so that we can do the experiment? On the other hand, if we don't do the exp. is it still okay to mention that GPL violation motivated us to do this work?}

When the source code of an executable or a library   is not available, binary code search can facilitate the tasks of GPL violation detection~\cite{DBLP:conf/msr/HemelKVD11}, software plagiarism detection~\cite{luo2014semantics,cop-tse}, reverse engineering~\cite{caballero2009binary}, semantic recovery~\cite{kim2014reuse}, malware detection~\cite{ming2015memoized} and buggy (vulnerable) code identification~\cite{DBLP:conf/sp/PewnyGGRH15,DBLP:conf/pldi/DavidY14,vuddy} in various software components.
However, binary code search is a fundamental problem that is much more challenging than the problem of source code search, due to the syntax gaps caused by the platform, the architecture and the compilation options.

%We can even search for zero-day vulnerabilities in proprietary binary by matching the known vulnerability from open source software.
%, in various software components for which the source code is not available (e.g., legacy application), have rise the demanded for scalable binary code searching techniques.
%recommend similar code solutions, identify binary semantics, and also reveal zero-day vulnerabilities. %Especially in software testing and security domain, vulnerabilities in software are deep hidden in inscrutable binary code, leaving a legacy ticking-time bomb in a software system.
%Especially, among the various vulnerability detection techniques~\cite{DBLP:journals/csur/ShahriarZ12}, binary code search is a fast yet accurate solution that preludes a further manual check of security experts.

In source code search, the similarity of two code segments is measured based on some representations of source code, e.g., approaches based on token~\cite{DBLP:journals/tse/KamiyaKI02}, abstract syntax tree (AST) \cite{DBLP:conf/icse/JiangMSG07}, control flow graph (CFG) \cite{DBLP:conf/qsic/ChanC14},  or program dependency graph (PDG)~\cite{DBLP:conf/icse/GabelJS08}. All these representations capture the syntactic or structural information of the program, and yield accurate results for source code search. However, these approaches fail when applied to binary code search. The reason is that the same source code may be compiled into the assembly code with different structures due to the choice of architecture, platform or compilation options. Thus, the syntactic or structural information is not sufficient for accurate binary code search.

%As summarized in our previous paper~\cite{bingo},
We propose three desirable characteristics for an accurate yet scalable cross-architecture and cross-OS binary code search tool.
%\vspace{-2mm}
\begin{itemize}%[label=\textbf{P\arabic*.},itemindent=*,itemsep=0mm]
\itemsep0em
\item \textbf{P1.} Resilient to the syntactic and structural differences  due to different configurations of compilers, architectures and OSs.
\item \textbf{P2.} Accurate in capturing the abstract and complete function semantics, balancing the impact of compiler optimization levels.
\item \textbf{P3.} Scalable to large-size real-world binaries, avoiding the overheads of the analysis based on dynamic executions or constraint solving.
\end{itemize}
%\vspace{-2mm}

%Recently, various approaches have been proposed to detect the similar binary code by using static or dynamic analysis.
%To better understand the mainstream binary function matching techniques,
The existing binary code search approaches adopt two types of analysis: namely, static analysis~\cite{DBLP:conf/pldi/DavidY14,saebjornsen2009detecting,luo2014semantics,DBLP:conf/sp/PewnyGGRH15} and dynamic analysis \cite{DBLP:conf/issta/JiangS09,DBLP:conf/asplos/Schkufza0A13,DBLP:conf/icse/JhiWJZLW11,egele2014blanket}.
Static approaches  usually  rely on the syntactical and structural information of binaries, especially CFG (i.e., a control flow of basic blocks within a function) to perform the matching. % , and does not capture the true semantics (i.e., complete understanding) of the program under investigation.
Dynamic approaches often inspect the invariants of input-output or intermediate values of program at runtime to check the equivalence of binary programs. In general, static approaches are good at P3, but suffer from P1 and P2. On the contrary, dynamic methods have merits in P2 and P3, while having drawbacks in P3.


%\begin{table*}[]
%\scriptsize
% \begin{center} %\vspace{-1mm}
%\caption{Comparison of existing techniques. Here, \ding{51} or \ding{55} represent whether it supports the corresponding feature or not, respectively. } \label{tab:lit_rev}
%\begin{tabular}{ | m{1.6cm}| m{1.1cm} | m{1.3cm} | m{1.3cm} | m{1.2cm} | }
%\hline
% \multicolumn{1}{|c|}{\textbf{Tool}} & \textbf{Type}  & \textbf{P1 Resilient}  & \textbf{P2 Accurate}  & \textbf{P3 Scalable}  \\ \hline
%   BLEX~\cite{egele2014blanket} & \multicolumn{1}{c|}{Dynamic} & \multicolumn{1}{c|}{\ding{51}} &  \multicolumn{1}{c|}{\ding{51}} & \multicolumn{1}{c|}{\ding{55}} \\ \hline
%  Tracy~\cite{DBLP:conf/pldi/DavidY14} & \multicolumn{1}{c|}{Static} & \multicolumn{1}{c|}{\ding{55}} & \multicolumn{1}{c|}{\ding{55}} & \multicolumn{1}{c|}{\ding{51}}  \\ \hline
%   \textsc{discovRE}~\cite{sebastian2016discovre} & \multicolumn{1}{c|}{Static} & \multicolumn{1}{c|}{\ding{55}} & \multicolumn{1}{c|}{\ding{55}} & \multicolumn{1}{c|}{\ding{51}}  \\ \hline
%     CoP~\cite{luo2014semantics} & \multicolumn{1}{c|}{Static} & \multicolumn{1}{c|}{\ding{55}} & \multicolumn{1}{c|}{\textcolor{black}{\ding{51}}{\small\textcolor{black}{\kern-0.55em\ding{55}}}} &\multicolumn{1}{c|}{\ding{55}}  \\ \hline
% Bug search~\cite{DBLP:conf/sp/PewnyGGRH15} & \multicolumn{1}{c|}{Static} & \multicolumn{1}{c|}{\ding{55}} & \multicolumn{1}{c|}{\ding{55}} &\multicolumn{1}{c|}{\ding{55}}  \\ \hline
%
%
% BinGo & \multicolumn{1}{c|}{Static} & \multicolumn{1}{c|}{\ding{51}} &\multicolumn{1}{c|}{\ding{51}} & \multicolumn{1}{c|}{\ding{51}}  \\ \hline
% BinGo-V & \multicolumn{1}{c|}{Static} & \multicolumn{1}{c|}{\ding{51}} &\multicolumn{1}{c|}{\ding{51}} & \multicolumn{1}{c|}{\ding{51}}  \\ \hline
%  \end{tabular}
%  \end{center}
%  \vspace{-4mm}
%\end{table*}


\begin{table*}[]
\scriptsize
 \begin{center} %\vspace{-1mm}
\caption{Comparison of existing techniques. Here, \ding{51} or \ding{55} represent whether it supports the corresponding feature or not, respectively. } \label{tab:lit_rev}
    \begin{tabular}{|l|m{1.2cm}||m{1.4cm}|m{1.4cm}|m{1.2cm}|m{1.2cm}||m{1.2cm}|m{1.2cm}|m{1.2cm}|m{1.2cm}|}
    \hline
    \textbf{Tool}  & \textbf{Type}  &  \textbf{Low-level semantic fea.} & \textbf{High-level semantic fea.} & \textbf{Structural fea.} & \textbf{Syntactic fea.} & \textbf{Constraint solving}& \textbf{Fast filtering}& \textbf{Selective inlining} & \textbf{Emulation} \\\hline
    \textsc{BLEX}~\cite{egele2014blanket}  & Dynamic  &  \multicolumn{1}{c|}{\ding{51}}      &   \multicolumn{1}{c|}{\ding{51}}     &   \multicolumn{1}{c|}{\ding{55}}     &     \multicolumn{1}{c||}{\ding{55}}  &  \multicolumn{1}{c|}{\ding{55}}    & \multicolumn{1}{c|}{\ding{55}} & \multicolumn{1}{c|}{\ding{55}}  & \multicolumn{1}{c|}{\ding{55}}  \\\hline
    \textsc{Tracy}~\cite{DBLP:conf/pldi/DavidY14} & Static &   \multicolumn{1}{c|}{\ding{55}}     & \multicolumn{1}{c|}{\ding{55}}       &   \multicolumn{1}{c|}{\ding{51}}     &   \multicolumn{1}{c||}{\ding{51}}     &  \multicolumn{1}{c|}{\ding{55}}     & \multicolumn{1}{c|}{\ding{55}} &  \multicolumn{1}{c|}{\ding{55}}   & \multicolumn{1}{c|}{\ding{55}} \\\hline
     \textsc{DiscovRE}~\cite{sebastian2016discovre}  & Static &  \multicolumn{1}{c|}{\ding{55}}      &  \multicolumn{1}{c|}{\ding{55}}      &  \multicolumn{1}{c|}{\ding{55}}      &      \multicolumn{1}{c||}{\ding{55}}  &  \multicolumn{1}{c|}{\ding{55}}    &\multicolumn{1}{c|}{\ding{51}}  &   \multicolumn{1}{c|}{\ding{55}}   &\multicolumn{1}{c|}{\ding{55}}   \\\hline
     \textsc{CoP}~\cite{luo2014semantics,cop-tse}   & Static &    \multicolumn{1}{c|}{\ding{51}}   &   \multicolumn{1}{c|}{\ding{55}}     &  \multicolumn{1}{c|}{\ding{55}}      &    \multicolumn{1}{c||}{\ding{55}}    &   \multicolumn{1}{c|}{\ding{51}}    & \multicolumn{1}{c|}{\ding{55}} &  \multicolumn{1}{c|}{\ding{55}}   &  \multicolumn{1}{c|}{\ding{55}} \\\hline
    Bug search~\cite{DBLP:conf/sp/PewnyGGRH15} & Static &   \multicolumn{1}{c|}{\ding{51}}    &   \multicolumn{1}{c|}{\ding{55}}     &  \multicolumn{1}{c|}{\ding{55}}      &      \multicolumn{1}{c||}{\ding{55}}  &  \multicolumn{1}{c|}{\ding{51}}     & \multicolumn{1}{c|}{\ding{55}} &  \multicolumn{1}{c|}{\ding{55}}   & \multicolumn{1}{c|}{\ding{55}}   \\\hline
     \textsc{Esh}~\cite{DBLP:conf/pldi/DavidPY16} & Static &  \multicolumn{1}{c|}{\ding{55}}      & \multicolumn{1}{c|}{\ding{55}}       &     \multicolumn{1}{c|}{\ding{51}}     &      \multicolumn{1}{c||}{\ding{55}}  & \multicolumn{1}{c|}{\ding{55}}     &\multicolumn{1}{c|}{\ding{55}}  & \multicolumn{1}{c|}{\ding{55}}     & \multicolumn{1}{c|}{\ding{55}}   \\\hline
    \textsc{\tool} & Static &  \multicolumn{1}{c|}{\ding{51}}      &  \multicolumn{1}{c|}{\ding{55}}      &  \multicolumn{1}{c|}{\ding{55}}      &   \multicolumn{1}{c||}{\ding{55}}     &    \multicolumn{1}{c|}{\ding{51}}  &\multicolumn{1}{c|}{\ding{51}}  &  \multicolumn{1}{c|}{\ding{51}}   & \multicolumn{1}{c|}{\ding{55}}   \\\hline
     \textsc{\toolNew} & Static &   \multicolumn{1}{c|}{\ding{51}}     &    \multicolumn{1}{c|}{\ding{51}}    &    \multicolumn{1}{c|}{\ding{51}}    &   \multicolumn{1}{c||}{\ding{55}}     &   \multicolumn{1}{c|}{\ding{55}}    & \multicolumn{1}{c|}{\ding{55}} & \multicolumn{1}{c|}{\ding{51}}    & \multicolumn{1}{c|}{\ding{51}}  \\\hline
    \end{tabular}%
  \end{center}
\vspace{-4mm}
\end{table*}

In Table~\ref{tab:lit_rev}, we list the state-of-the-art binary code search tools in the literature. Among these tools, only \textsc{\small BLEX}~\cite{egele2014blanket} is the dynamic function matching tool that uses seven semantic features generated from running execution. %\textsc{\small BLEX}~\cite{egele2014blanket} could be accurate, but not scalable.
All the other tools adopt static analysis, and hence suffer from the issues in P1 and P2. \textsc{\small Tracy}~\cite{DBLP:conf/pldi/DavidY14} uses the pure syntax information for function matching, and it uses $k$-tracelet (i.e., basic blocks of length $k$ along the control-flow path), which is architecture- and OS-dependent. \textsc{\small discovRE}~\cite{sebastian2016discovre} proposes to find across-architecture bugs in binaries in a scalable manner, where it uses two filters (numeric and structural) to quickly locate the similar candidates for the target function. \textsc{\small CoP}~\cite{luo2014semantics} is a plagiarism detection tool that adopts the theorem prover to search for semantically equivalent code segments.
%, hence, not very scalable for real-world binaries \mahin{and does not support cross-architecture and cross-OS analysis}.
The bug search tool proposed in~\cite{DBLP:conf/sp/PewnyGGRH15} supports cross-architecture analysis  by translating the binary code to an intermediate representation, and solving the resulting assignment formulas of input and output variables.
\textsc{Esh}~\cite{DBLP:conf/pldi/DavidPY16}, inspired by the idea of similarity by the composition for images, proposes  to  decompose each procedure to small code segments (so called \emph{strands}), semantically compare the strands to identify
similarity, and lift the results into procedures.

%Since different OSs use different ABI (Application Binary Interface), \cite{DBLP:conf/sp/PewnyGGRH15} has very limited support for cross-OS analysis because of ignoring ABI in the analysis.
%Since extracting semantic features is quite expensive and there is no pre-filtering process in place to narrow down the search space,  \cite{DBLP:conf/sp/PewnyGGRH15} is not very scalable and in addition, due to incomplete modeling, it captures only the partial semantics of the functions.
%ABI in the analysis.
%Further, \textsc{\small CoP} and~\cite{DBLP:conf/sp/PewnyGGRH15} are program structure dependent, where they use pairwise basic-block similarity search as an initial step to identify candidate target functions that are similar to the signature\footnote{\mahin{define signature and target functions}}. This indicates that both \textsc{\small CoP} and~\cite{DBLP:conf/sp/PewnyGGRH15} have an implicit assumption that at least one basic-block is preserved in the signature and target binaries. These assumptions are too restrictive for real-world binaries especially, when the signature and target binaries (or fucntions) do not share the \mahin{same code base}.

  %, which captures the precise semantics of the binary code.
%As a dynamic analysis tool, \textsc{\small BLEX} %does code search at the function level
%is not scalable and, due to implementation limitations, does not support cross-architecture and cross-OS analysis.


%\begin{table}[]
%\scriptsize
%	\begin{center} \vspace{-1mm}
%\caption{Comparison of existing techniques} \label{tab:lit_rev}
%\begin{tabular}{ | m{1.6cm} | m{2.2cm} | m{2.1cm} | m{0.9cm} | }
%\hline
%	\textbf{Tool}  & \textbf{Program structure neutral (C1)} & \textbf{Capture complete semantics (C2)} & \textbf{Scalable (C3)} \\ \hline
%	 BLEX~\cite{egele2014blanket} & Yes &  Yes & No \\ \hline
%	 Tracy~\cite{DBLP:conf/pldi/DavidY14}  & No & No & Yes  \\ \hline
%     CoP~\cite{luo2014semantics} & No & Limited & No  \\ \hline
%	Bug search~\cite{DBLP:conf/sp/PewnyGGRH15} & No & No & No  \\ \hline
%	discovRE~\cite{sebastian2016discovre}& No & No & Yes  \\ \hline
%	 BinGo & Yes (\S\ref{sec:func_match}) & Yes (\S\ref{sec:inline}) & Yes (\S\ref{sec:prefilter}) \\ \hline
%  \end{tabular}
%  \end{center} %\vspace{-4mm}
%\end{table}

%\begin{table*}[t]
%\scriptsize
%	\begin{center}
%\caption{Comparison of existing techniques} \label{tab:lit_rev}
%\begin{tabular}{ | m{2.1cm} |  m{1.2cm} | m{4cm} | m{2cm} | m{1.3cm} | m{2.5cm} |  m{1.0cm}| }
%\hline
%	\textbf{Tool} & \textbf{Technique} & \textbf{Similarity matching} & \textbf{Cross-architecture} & \textbf{Cross-OS} & \textbf{Complete  semantics} &  \textbf{Scalable} \\ \hline
%	 BLEX~\cite{egele2014blanket} & Dynamic & Whole-function matching &  Limited & Limited & Yes & No \\ \hline
%	 Tracy~\cite{DBLP:conf/pldi/DavidY14} & Static & Partial trace (fixed length) matching & No & No & No & Yes \\ \hline
%     CoP~\cite{luo2014semantics} & Static & Pairwise basic-block matching& No & No & No & No \\ \hline
%	Bug search~\cite{DBLP:conf/sp/PewnyGGRH15} & Static & Pairwise basic-block matching & Limited & Limited & No & Yes \\ \hline
%	discovRE~\cite{sebastian2016discovre} & Static & Pairwise basic-block matching & Limited & Limited & No & Yes \\ \hline
%	 BinGo & Static & Partial trace (variable length) matching & Yes & Yes & Yes &  Yes \\ \hline
%\end{tabular}
%\end{center} \vspace{-7mm}
%\end{table*}
% followed by LCSSEBB\footnote{Longest common subsequence of semantically equivalent basic-block}
%followed by BHB\footnote{Best-Hit-Boarding algorithm


%Here, true semantics of a program (or function) refers to the complete understanding of the functionality of the program.
%\textsc{\small BLEX}~\cite{DBLP:conf/uss/EgeleWCB14} is the latest tool in this line, as it uses seven semantic features from an execution (e.g., calling imported library functions). %However, \textsc{\small BLEX} is for x64 binaries, not for cross-architecture code search. Pewny \emph{et al.} \cite{DBLP:conf/sp/PewnyGGRH15} propose a purely static analysis that can detect the similar code among  the binaries of the application on different OSs. This is good for finding clones of the same program due to architecture or compilation differences, but maybe not good at finding relaxed binary clones among different applications.

%These approaches can be effective, but they face challenges from two aspects: the difficulty in setting up the execution environment to dynamically execute, and the scalability issue that prevents large-scale analysis.
%In the following, we highlight the key limitations in the existing binary function matching techniques.
%Table~\ref{tab:lit_rev} shows that static approaches cannot precisely capture the full semantics, while dynamic methods struggle for the scalability.
%To have a better comparison, we list the key challenges that restrict the existing approaches.

%\mahin{In all the techniques listed in Table~\ref{tab:lit_rev}, it is implicitly assumed that both the signature and target binaries share (or forked from) the same code base. However, this assumption is too restrictive and have limited real-world applications \footnote{This assumption is justifiable for only \textsc{\small CoP} as it is a plagiarism detection technique}.}
%\mahin{In the techniques presented in Table~\ref{tab:lit_rev}, we find three key limitations and they are summarized below:}

%\renewcommand{\theenumi}{\arabic{enumi}}

%\ly{I am not happy with the following paragraph, let's discuss tmr.}

In Table~\ref{tab:lit_rev}, we list the major features\xyx{\footnote{Refer to section \ref{sec:category} for details of each category of features.}} used for similarity scoring and the search optimization techniques in these tools. \emph{Low-level semantic fea.} refer to the features of low level architectural information used for matching (e.g., symbolic analysis on the CPU flag and register values~\cite{bingo}). \emph{High-level semantic fea.} refer to the features of API-relevant information used for matching (e.g., the names and sequence of invoked built-in APIs of a lib \cite{egele2014blanket}).
 \emph{Structural fea.} refer to the structural information used by the static approaches (e.g., basic-block structures used in~\cite{luo2014semantics,DBLP:conf/sp/PewnyGGRH15,sebastian2016discovre} and fixed length tracelet in \cite{DBLP:conf/pldi/DavidY14}).  \emph{Syntactic fea.}  refer to the syntactic information used for matching (e.g., the function name, \xyx{or string literals extracted from binary \cite{DBLP:conf/msr/HemelKVD11}}).

In general, merely relying on structural and syntactic features cannot make the match across the boundary of architecture, OS or compiler options, and hence fail P1. Instead, semantic features help to achieve P1. Even with the semantic features, the existing static  approaches may not capture the complete function semantics due to different function inlining options of candidate functions~\cite{DBLP:conf/sp/PewnyGGRH15}. %For example, the ability of \textsc{\small CoP}~\cite{luo2014semantics} to capture semantics is limited to the \xyx{inner-procedure} execution path only. \mahin{we may not include this. because, theoretically CoP can do this but they didn't demonstrate it in the paper. I need to think about a proper example}.
Hence, the proper inlining strategy is desirable for achieving P2. Besides, scalability is a real challenge for tools \cite{luo2014semantics,DBLP:conf/sp/PewnyGGRH15} based on symbolic analysis (low-level semantic features) and constraint solving. For example, \textsc{CoP} took half a day to find target functions in Firefox for a few signature functions~\cite{luo2014semantics}. Thus, to achieve P3, time-consuming constraint solving should be avoided.

%It can be seen easily that static approaches are not precise, while dynamic methods struggle for the scalability.

%, the above challenge are the keys factors .

%\textbf{P1} requires the solution to be architecture- and OS- neutral, and be resistant to the variances that arise due to the differences in compiler type and optimization level.
%Static techniques based on the structural properties of binary functions will fail~\cite{DBLP:conf/pldi/DavidY14,luo2014semantics,DBLP:conf/sp/PewnyGGRH15,sebastian2016discovre}.
%Most of the static analysis also assume that the functions are self-contained (i.e., invocations of libraries and other user-defined functions are ignored). Thus the true function semantics are not fully captured~\cite{DBLP:conf/pldi/DavidY14,DBLP:conf/sp/PewnyGGRH15, sebastian2016discovre}.
%Lastly, scalability still requires more efforts to avoid matching every single target function in the database~\cite{egele2014blanket,DBLP:conf/pldi/DavidY14,luo2014semantics,DBLP:conf/sp/PewnyGGRH15}.

%This work aims at addressing the limitations to build a robust, scalable binary code searching (or function matching) tool that can support cross-architecture, OS and compiler (with various optimization levels) analysis with less (no) assumptions on the nature of signature and target binaries.  To this end, we design an approach to address the aforementioned problems with the following merits.

In our previous work \cite{bingo}, we propose a binary code search tool, named \tool.
For P2, \tool adopts a \emph{selective inlining} (\S\ref{sec:inline}) to include relevant libraries and user-defined functions in order to capture the complete semantics of the function. For P3, a filtering technique   is proposed to shortlist the candidate functions. Subsequently, after inlining and filtering, for P1, \tool extracts the partial traces of various lengths from the candidate functions as function models, and then extracts \emph{low-level semantic features} from the function models (\S\ref{sec:category:lowSeFea}). Last, \tool measures the function similarity score by Jaccard containment similarity of features.



%Technically, first, to recover the  complete semantics from the functions under investigation, we propose a \emph{selective inlining} technique, where the callee (both libraries and user-defined) functions are inlined into the caller such that the complete function semantics are captured~\cite{wang2015binary} (\S\ref{sec:inline}).
%To avoid code size explosion, we selectively inline the callee functions based on the invocation dependency patterns, which differs from the traditional compiler inlining optimization techniques for maximum speed or minimum size~\cite{chang1992profile}.
%To our best knowledge, this work is the first attempt towards investigating selective inlining in recovering binary semantics.
%Second, to improve the scalability of our approach, we propose an \emph{architecture and OS neutral filtering} technique that narrows down the search space by shortlisting the candidate target functions for binary semantic matching (\S\ref{sec:prefilter}).
%Next, to overcome the limitation of basic-block structures (\S\ref{sec:func_match}), we generate \emph{function models}, which are agnostic to the underlying program structure, via the length variant partial traces\footnote{A partial trace refers to a sequence of basic-blocks that lie along an execution path in the CFG~\cite{DBLP:conf/pldi/DavidY14}.} (called, partial traces of \emph{k} lengths). For each function, partial traces of length $1$ to $k$ are extracted to form the function model such that it represents the function at various granularity levels.
%Here, we also take measure to minimize the effects of infeasible paths and compiler specific code in calculating the function similarity scores.
%Finally, semantic features are extracted from the function models of candidate target functions, for function similarity scoring, where semantic features capture the machine state transitions in the form of Input/Output pairs (\S\ref{sec:func_match}).

In our previous evaluation on a number of real-world binaries \cite{bingo}, \tool outperforms the available state-of-the-art tools (i.e., \textsc{Tracy}~\cite{DBLP:conf/pldi/DavidY14} and \textsc{BinDiff}~\cite{DBLP:conf/dimva/Flake04}),  for the same tasks in terms of search accuracy and performance. %Applying \tool, we also discovered a 0day vulnerability (CVE-2016-0933) in the Adobe PDF Reader with the 4000USD bug bounty.
However, we found some false positive (FP) cases in using \tool. When a function has a small size of assembly instructions, the low-level semantic features extracted from this function could be too generic to contain any real semantics. For example, \xyx{Fig. \ref{fig:falseposi} shows a false positive case of \tool~due to the same input/output values (see details in \S \ref{subsec:sem_chall_sol})}. %\mahin{I'll need to find a suitable example.}

To address the FP cases, the new version of our tool, \toolNew, supports high-level semantic features and structural features. High-level semantic features we use are the information of system calls or library calls (\S \ref{sec:category:highSemanticFea}). The rationale is that such information is commonly used for malware detection \cite{DBLP:conf/issta/CanaliLBKCK12}, as they can reveal the semantics of the functions. On the other hand, we borrow the idea of fast bytecode clone detection based on the approach of projecting the CFG  into a  centroid of 3 dimension coordinate \cite{DBLP:conf/icse/ChenLZ14}. Inspired by this approach (3D-CFG for short \cite{DBLP:conf/icse/ChenLZ14}), the problem of graph isomorphism is reduced to a problem of calculating the distances among centroids. 3D-CFG implemented in  \toolNew adopts the following structural information (\S \ref{sec:category:structralFea}): \xyx{basic block (BB) sequence, loop information, and in-degree/out-degree of BB. }

To speed up the process, \toolNew leverages on emulation rather than  constraint solving that is used in extracting low-level semantic features. %Note that we refer \emph{micro execution} to the term of the mechanism to execute any assembly code snippet without test driver or user input, not to the tool \textsc{MicroX}~\cite{DBLP:conf/icse/Godefroid14}. \textsc{MicroX}, Microsoft's VM tool allowing micro execution of x86 binary code, cannot be used for cross architecture code search. Instead,
We adopt \textsc{Unicorn}~\cite{unicorn}, a lightweight multi-platform, multi-architecture CPU emulator framework, to virtually execute the partial traces that are extracted from the function models of a given function. The emulation step may take significantly less time than constraint solving. The challenge of integrating \textsc{Unicorn}~\cite{unicorn} lies in \xyx{the handle of function calls (\S \ref{sec:emulation})}. Last, similarity scores of two functions in terms of structures, low-level/high-level semantics are combined in the final matching (\S \ref{sec:func_match}).

\noindent\textbf{Key contributions.} Beyond the contributions that have been made in our previous work \cite{bingo}, we make these  new contributions:
\begin{itemize}[nolistsep]
\itemsep0em
\item Our work is the first attempt to incorporate emulation with cross-architecture cross-OS binary code search.
 %We leverage on \emph{k}-length partial traces to model the function at various granularity levels that is agnostic to underlying program structures.
\item We extract structural features based on the idea of centroid based clone detection. Previously, 3D-CFG based matching was mainly applied for bytecode clone detection.

 \item \xyx{We combine different categories of features listed in Table \ref{tab:lit_rev}. We evaluate our approach under the scenarios of cross-architecture matching, cross-OS matching, and cross-compiler matching.}
%\item introduce an architecture and OS neutral function filtering process that helps narrow down the target function search space.
\item We empirically compare \toolNew with \tool and the available state-of-the-art tools of binary code search in terms of search accuracy and performance. %We also apply \toolNew for the task of library function identification. The results show that \toolNew  can improve XX\% of accuracy  compared with the state-of-art tool REFG \cite{DBLP:journals/tse/QiuSM16}.  %we also show that \tool can be used to hunt zero-day vulnerabilities from COT binaries.
\end{itemize}


%aum ganathipathaye namaha
%sri rama jeyam

%aum ganathipathaye namaha
%sri rama jeyam


\noindent\textbf{Experimental results.} \xyx{We conduct various experiments to evaluate the effectiveness of \toolNew. For cross-architecture matching on \texttt{coreutils} binaries,  on average, the rate of best match for \toolNew (65.6\%) is significantly better than that of \tool (35.1\%) . For cross-compiler matching and intra-compiler matching on \texttt{coreutils} binaries, on average, the rate of best matching is improved from the range of 30\%-60\% (for \tool) to the range of 70\%-99\% (for \toolNew). For cross-OS matching on Windows binary \texttt{mscvrt} and Linux binary \texttt{libc}, on average, the rate of best matching is improved from 22\% (for \tool) to 51.7\% (for \toolNew). In additional experiments on binary code matching on forked projects, \toolNew achieves a higher rate of best matching (93.6\%) than that (88\%) of the existing benchmark tool. Last, by virtue of emulation, the most time-consuming part of low-level semantic feature extraction is significantly accelerated, which allows us to introduce more features to improve the accuracy. }

\begin{figure}[t]
  \includegraphics[width=\linewidth]{srj-figures/srj-moti_ex2.pdf}
  \caption{Code segment responsible for Heartbleed vulnerability (CVE-2014-0160) appeared as in the binary (a) compiled with \texttt{gcc} 4.6, and (b) compiled with \texttt{mingw}32} \label{fig:prob_stat}
% \vspace{-4mm}
\end{figure}


\begin{figure*}[th]
  \centering
  \subfigure[String copy via calling function \emph{strcpy}]{
    \label{fig:falseposi:a} %% label for first subfigure
    \includegraphics[width=1.8in]{srj-figures/bingoV_exam1a.png}}
  \hspace{1in}
  \subfigure[Memory copy in an iterative way]{
    \label{fig:falseposi:b} %% label for second subfigure
    \includegraphics[width=3.82in]{srj-figures/bingoV_exam1b.png}}
  \caption{A false positive case of \tool}
  \label{fig:falseposi} %% label for entire figure
\end{figure*}

\section{Motivation} \label{sec:prob_state}
%Identifying semantically similar or equivalent binary code is a challenging task.
%In this section, %we discuss the problems of the existing binary code search techniques with motivating examples.
%Then
In this section, we state the challenges in the current research of cross-architecture and cross-OS binary code search  with illustrative examples. %We briefly introduce the solutions to these challenges, by leveraging selective inlining, combining different categories of features and adopting emulation.


 %for complete semantics analysis of functions and necessity to have a function model that is agnostic to underlying program structure.
%Then, we introduce the key challenges faced by existing binary function matching tools in detail.

%we \xyx{give a motivating example of binary matching regardless of architecture and OS differences, and state the challenges of  binary code search in finding the semantics and structures for this example. Last, we explain the basic idea of our proposed solution. }%and discuss their roles to achieve the scalable binary matching regardless of architecture and OS differences.}

\subsection{Motivating Example}  \label{subsec:bin_pre}

A binary program consists of a number of functions, each of which can be represented by the CFG (control-flow graph), a (directed) graph of basic blocks (BBs). The assembly instructions in a function are systematically  grouped into several BBs, which are considered as the building blocks of binary program. Thus, this BB-based representation is used by many static binary analysis tools.
%\begin{mydef}
%\emph{(\textbf{Basic-block}) A sequence of assembly instructions without any jumps or jump targets in the middle, where a jump target starts a block, and a jump ends a block.}
%\end{mydef}

%\begin{example}
 The same source code may produce binary code of different BB structures, due to different compilation configurations. Taking the heartbleed vulnerability (CVE-2014-0160) for example, Fig.~\ref{fig:prob_stat}(a) and (b) show the BB structures of the same source code compiled with \texttt{gcc} and \texttt{mingw}, respectively. Apparently, these two binary code segments share no identical BB structures --- with \texttt{gcc}, the vulnerable code is represented as a single basic block;  with \texttt{mingw}, represented as several basic blocks. A detailed inspection suggests that  library function \texttt{memcpy} is inlined in \texttt{mingw} version; while in \texttt{gcc}, it is not. %Due to this library function inlining, in \texttt{mingw}, the similarity between basic-blocks is affected considerably.
%Given the \texttt{gcc} version as a signature, we may miss the vulnerability in \texttt{mingw} version due to the function inlining.
Huge differences between the binaries in syntax and program structure make binary code search a challenging task.
%Hence, basic-block centric
%function modeling similarity matching(\textbf{P1}) and \mahin{program structure dependent pre-fitlering (\textbf{P3})} is not robust enough to address real-world problems. Further, this real-world scenario strongly suggest that functions should not be analysed in isolation, i.e., to capture the true semantics of a functions (\textit{caller}), all the related functions (\textit{callee}) also need to be taken in to consideration based on the caller-callee relationship (\textbf{P2}).
%\end{example}

 % and it is formally defined as follows (adapted from~\cite{david2014tracelet}):
%
%\begin{mydef}
%\emph{(\textbf{Type-k Partial Trace\footnote{In the rest of the sections, we'll use the term `partial trace' and `type-k partial trace' interchangeably, where `type-k partial trace' is used when referring to the length of a partial trace and in all other instances the term `partial trace` is used}}) Type-k partial traces is an ordered tuple of $k$ sequences, each representing one of the basic-blocks in a directed acyclic sub-path in the CFG, and containing all of the basic-block's assembly instructions.
%}
%\end{mydef}



%We introduce the term \emph{environment}, where binary is generated and executed, to denote the underlining architecture (e.g., Intel, ARM), the operating systems (OS) (e.g., Windows, Linux), the used compiler, and also the chosen compilation options.
%Different architectures have different instructions (a.k.a. ISA or Instruction Set Architecture) for the machine.
%Obviously, different environments will lead to considerably different binaries.

%\note{\textbf{I think we need make sure environment is used in the rest of the sections}}


%To better find the assembly code clones that perform similarly computational tasks, we formulate the concept of semantic clones.

% which has very important applications in both software security and software engineering. Despite the challenges, in the recent year, there are several good solutions proposed in academia to tackle this problem~\cite{pewnycross,ruttenberg2014identifying,egele2014blanket,luo2014semantics}. %of identifying semantically equivalent functions at binary level.
%However, none of the existing studies fills up the theory blank of \lq{}how a function is modelled\rq. %In the proposed modelling techniques of {\color{red}{using XXX}}~\cite{pewnycross,luo2014semantics}, it is assumed that basic-block structure is preserved across binaries, and thus, functions models should be  basic-block centric. Based on this assumption, semantic features are extracted from the sample functions at basic-block level and compared with the counterparts extracted from target functions in a pairwise way of basic-block comparison.


\subsection{Challenges for Syntactic Approaches}\label{sec:back:challenge}
Syntax is the most direct information usable for code search.
%However, the key challenge for syntax-based matching is:
%\noindent \textbf{C1: There is no consistent binary syntax representation across architectures.}
%Existing approaches mostly have attempted to use instruction patterns~\cite{DBLP:conf/uss/JangWB13,DBLP:conf/raid/KrugelKMRV05,saebjornsen2009detecting,DBLP:conf/pldi/DavidY14}, where for similarity matching, these approaches rely on syntax information. Though these techniques work well for the given environment (e.g., cases presented in~\cite{DBLP:conf/pldi/DavidY14}), it  fails for cross-architecture analysis, where syntax dramatically changes for different architectures~\cite{DBLP:conf/sp/PewnyGGRH15}.
%%For example, the two binary code segments in Fig. \ref{fig:idiom2-ex}, for ARM (left) and x86 32bit (right) architectures,  both represent the stack frame set up operation in function prologue.
%Hence, it is hard to perform similarity matching across architectures by purely relying on the syntax representation.
Most of existing approaches that rely on syntax information have attempted to use instruction patterns~\cite{DBLP:conf/uss/JangWB13,DBLP:conf/raid/KrugelKMRV05,saebjornsen2009detecting,DBLP:conf/pldi/DavidY14}.
As no consistent low-level syntax representation (i.e., assembly instructions) is available for cross-architecture search, these approaches fail for cross-architecture analysis. %\cite{DBLP:conf/sp/PewnyGGRH15} and CoP~\cite{luo2014semantics} use semantic features (e.g., symbolic expressions) and bug signatures to do cross-architecture analysis, respectively.}
To make binary code search across the architecture and OS boundary, semantics-based matching has been proposed~\cite{luo2014semantics,DBLP:conf/sp/PewnyGGRH15}, in which the machine state transition represents the semantics of a binary. Still, three challenges are faced in semantics-based matching.
%For example, the two binary code segments in Fig. \ref{fig:idiom2-ex}, for ARM (left) and x86 32bit (right) architectures,  both represent the stack frame set up operation in function prologue.

%\begin{figure}[ht]
%\small
% \centering
%  \begin{subfigure}[b]{0.5\linewidth}
%   % \includegraphics[width=0.75\linewidth]{srj-figures/srj-hierarchy-2.png}
%    \raggedright{\texttt{
%    \\
%    	mov ip, sp\\
%    	stm sp!, {fp,ip,lr,pc}\\
%    	sub fp, ip, 16\\
%    }}
%  %  \caption{\small{ARM version}}
%   \label{fig:idiom2-ex-a}
%    \vspace{1ex}
%  \end{subfigure}%%
%   \begin{subfigure}[b]{0.5\linewidth}
%   \centering
%   % \includegraphics[width=0.75\linewidth]{srj-figures/srj-hierarchy-2.png}
%    \raggedright{\texttt{
%    \\
%    	push ebp\\
%    	mov ebp,esp\\
%    	sub esp,16\\
%    }}
% %   \caption{\small{x86 32bit version}}
%   \label{fig:idiom2-ex-b}
%    \vspace{1ex}
%  \end{subfigure}%%
%  \caption{Function prologue for ARM (left) and x86 32bit (right)}
%  \label{fig:idiom2-ex}
%\end{figure}

\subsection{Challenges for Semantics-based Matching} \label{subsec:sem_chall}



\noindent\textbf{C1: The trade-off challenge of deciding the granularity of function model.} %: breaking the limits of basic-block structure.}
The state-of-the-art tools~\textsc{CoP}~\cite{luo2014semantics} and bug search \cite{DBLP:conf/sp/PewnyGGRH15} assume that BB-structure is preserved across binaries, and \xyx{a single BB can be matched with other BBs of similar semantics}. Based on BB-structure, semantic features are extracted and built into the function model. However, as admitted by Pewny \emph{et al.}~\cite{DBLP:conf/sp/PewnyGGRH15}, this approach is too sensitive to BB-structure differences, and hence problematic for smaller functions whose CFG structures are more susceptible to compilation options.
%Based on this, semantic features extracted from the signature function at basic-block level are compared with the counterparts extracted from target functions in a pairwise way.
%\emph{"Our metric is sensitive to the CFG and the segmentation of the basic block, which we found to be potentially problematic especially for smaller functions."}
%In practise, the assumption is too restrictive to be applied for real-world cases.
For example, the BB-structure compiled with  \texttt{gcc} 4.6 in  Fig.~\ref{fig:prob_stat}(a) is significantly different from that compiled with \texttt{mingw}32 in Fig.~\ref{fig:prob_stat}(b). In this case, the approach of BB-centric matching fails as no two BBs can match in Fig.~\ref{fig:prob_stat}.        % semantic features extracted from a single BB in these two cases will not contain the same semantics.





\noindent\textbf{C2. The accuracy challenge of using low-level semantic features.} %: covering the high-level and complete function semantics.}
Functions are considered in isolation by most of static tools, i.e., semantics of callee functions are not considered as part of the caller's semantics. This leads to partial semantics problem, especially when some common utility functions are implemented by the developers themselves (e.g., Adobe Reader has its own \texttt{malloc} implementation).
%Some existing techniques~\cite{wang2015binary} has noticed	the problem of basic-block, and propose to
\textit{Blindly inlining}  all the callee functions can be a remedy, as  all the user-defined functions are inlined in~\cite{wang2015binary}. %The blindly inlining strategy would lead to such a problem of code size explosion that the bloated functions are hard to analyze, incurring heavy overhead.
%However, one might assume that inlining all the callee functions at their respective call sites, as in~\cite{wang2015binary}, would solve the partial (or incomplete) semantics problem present in existing binary function matching techniques.
 Nevertheless, this approach fails in practice due to two main reasons: (1) heavy inlining may lead to code size explosion~\cite{chang1992profile},  %where performing similarity matching on bloated functions incur heavy overhead and hence,
and (2) not all the callee functions are semantically relevant to the caller function. %hence, inlining such functions might dilute the core functionality of the caller function, which in return, leads to poor matching results.
Thus, an inlining strategy is needed to decide that: \texttt{memcpy} should be inlined in Fig.~\ref{fig:prob_stat}(a) for matching with the semantically relevant one in Fig.~\ref{fig:prob_stat}(b). On the other side, \xyx{for functions not to be inlined, e.g., \texttt{ss13\_write\_bytes} in Fig.~\ref{fig:prob_stat}(a), features of library call information should not be ignored.}


\noindent\textbf{C3. The performance challenge of using static symbolic analysis or dynamic execution.} %: replacing with by micro execution.}
 Syntax based techniques, in general, are scalable~\cite{DBLP:conf/pldi/DavidY14}. As discussed earlier, they fail on cross-architecture analysis. To address this problem, semantics based approaches are preferred. However, extracting and solving low-level semantic features is a time-consuming job. Previously, we adopted an \emph{efficient} and architecture-, OS- and compiler-\emph{neutral} function filtering mechanism in \cite{bingo}. \xyx{With such mechanism, extracting input/output values of registers and flags of a function via constraint solving still takes 4,469s on average to extract low-level semantic features from 2,611 Linux \texttt{libc} functions --- it takes 1.7s to extract semantic features from a \texttt{libc} function. Hence, this part still has much space for improvement. To scale up code search for large-size binaries in real world, an efficient method is desired.}

\xyx{Previously, \tool has adopted $k$-tracelet for C1, selective inlining for C2, and filtering mechanism for C3 \cite{bingo}. However, \tool still suffers issues of false positive cases (see Fig.~\ref{fig:falseposi} and \S\ref{subsec:sem_chall_sol}) and inefficient feature extraction. In this paper, \toolNew adopts new features and emulation to address these issues (see \S\ref{sec:category}).}

%Therefore, to facilitate scalable semantic matching, an \emph{efficient}, and architecture, OS and compiler \emph{neutral} function filtering step is required.
%However, exiting approaches focus only on the efficiency aspect of the filter. For example, in~\cite{sebastian2016discovre}, only program structural features, such as number of basic blocks, are used as filters to speed up. However, as shown in Fig.~\ref{fig:prob_stat}, such features are not robust enough for cross-architecture, OS and compiler analysis.

%For example, due to the simplicity of the code in Fig.~\ref{fig:prob_stat}(a), there are many semantically irrelevant functions, which have the same number of basic blocks (i.e., one), to be filtered out before semantic matching.

%
%\noindent\textbf{C2: Semantics of binary code is too low level.}
%Due to the lack of explicit program semantics at binary level, machine state transitions of two totally unrelated code segments may look very similar (or identical), if the number of instructions in those binary segments are too small, e.g., the two code segments in Fig.~\ref{fig:code_seg}.
%A binary modeling approach based on machine state transitions has the problem of producing high false positives, especially when the given function (or vulnerability) signature is too small~\cite{DBLP:conf/sp/PewnyGGRH15}.
%%\ly{combine fig 2 and 3?}
%%Unfortunately, signatures, especially vulnerability signatures, having few instructions is common in real-world scenarios~
%
%\begin{figure}[t]
% \begin{center}
% \subfigure[A code segment to XX]{\includegraphics[width=200pt]{srj-figures/runexample1.pdf}}\label{fig:seg1:a}
% \subfigure[A code segment to XX]{\includegraphics[width=200pt]{srj-figures/runexample2.pdf}}\label{fig:seg1:b}
%\end{center}
%\vspace{-4mm}
%\caption{\toolNew system architecture}
%\label{fig:code_seg}
%
%\end{figure}

%
%As shown in Fig. \ref{fig:state_seg}, two unrelated segments in Fig.~\ref{fig:code_seg} have the same initial machine state (i.e., pre-state) and also the identical post-state after execution. %, both code segments will result in .
%Here, code segment 1 adds the values in $\mathtt{EAX}$ and $\mathtt{EBX}$ registers, and move the results to $\mathtt{EAX}$ register. Since the pre-state values are all set to zero, after execution, $\mathtt{EAX}$ register will hold the value 0 and the condition-code flag $\mathtt{ZF}$ (zero flag) will be set to 1, since the result of addition operation is zero. On the other hand, in code segment 2, the value in $\mathtt{EBX}$ register is pushed to the stack, system API $\mathtt{strlen}$ is invoked, and finally the return value that is stored in $\mathtt{EAX}$ register is compared with an immediate value `0'. After execution, similar to code segment 1, $\mathtt{EAX}$ register will hold the value 1 as $\mathtt{strlen}$ system API is not interpreted (e.g., \cite{DBLP:conf/sp/PewnyGGRH15,luo2014semantics}), $\mathtt{EAX}$ register will hold the pre-state value, which is 0. In addition, the comparison operation will set the $\mathtt{ZF}$ to 1. In both executions, none of the other registers, condition-code flags and memory locations will be modified, hence retaining the pre-state values. From the examples shown in Fig.~\ref{fig:code_seg} and~\ref{fig:state_seg}, it can be seen that machine state transitions alone are not sufficient to model binary code.
%%\end{example}

%\noindent \textbf{C3: Computing the precise semantics is difficult and expensive.}
%Static methods~\cite{luo2014semantics,DBLP:conf/sp/PewnyGGRH15} using state-based semantic computation cannot capture the precise semantics due to missing the consideration of system APIs in the analysis.
%Dynamic analysis based techniques such as~\cite{egele2014blanket} might be able to address this problem. However, it is not scalable enough to handle large binaries. For example, in \cite{egele2014blanket}, before each function is executed, an environment needs to be set so that the execution does not terminate due to uninitialised memory handling. Unfortunately, considering the fact that a moderate size binary would easily have several thousand functions, this approach is not very scalable. %Hence, to address these challenges, we propose a partial traces based function modelling technique.







%\begin{mydef} \label{def:tracelet}
%\emph{\textbf{A partial trace} is \note{an instruction sequence obtained from basic-blocks that lie adjacent to each other along a program execution path. Partial traces can be of different length, where partial trace of length $k$ (called, Length-$k$ partial trace) denotes a partial trace that contains the instructions obtained from $k$ adjacent basic-blocks that lie along a program execution pat }}
%\end{mydef}
%
%\begin{mydef} \label{def:comp_semantic}
%\emph{\xyx{\textbf{Complete (or true) Semantics} refer to the contextual semantics that the function under analysis and %why we need true semantics of a function because compiler inline/outline certain functions
%}}
%\end{mydef}


%As discussed in Section \ref{sec:prob_state},
%The existing static binary function matching techniques, in general, fail to take into account the true (or complete) semantics of the functions under investigation. That is, each function in a binary is analyzed in isolation, where the semantics of the callee functions (be it user-defined or library function) are not considered when generating the caller function semantics as they are assumed to be two different functions. However, this assumption is intuitive and valid until we understand the caller-callee relationship (or dependency). For example, assume a user-defined function that manipulates string literals invokes the \texttt{strcpy} library function, and in order to capture the true semantics of the caller, it is \note{essential?} to inline \texttt{strcpy} function at the call site, as the caller function is involved in string manipulation, where it leverages on utility fucntions, such as \texttt{strcpy}, provided by the C runtime library to simplyfy its operations, hence, the semantics of the utility functions need to be considered as part of the caller fucntion semantics. Similarly, in another occasion, the programmer might implement her own version of \texttt{strcpy} function with additional security properties (called, \texttt{strcpy\_secure}\footnote{It is worth noting that \texttt{strcpy} is one of the banned function calls by Microsoft and hence, its better to use more secure variants such as \texttt{strncpy\_s} and \texttt{strcpy\_s}~\cite{msbannedfunctioncalls}}) and invokes it from all the string manipulation functions that need a secure string copy operation, hence, to capture the true caller semantics, the user-defined \texttt{strcpy\_secure} function needs to be inlined at the call site.

%To this end, in this paper, we propose partial trace based function modelling to overcome the aforementioned limitation.

%\noindent\textbf{$3$-tracelet or $k$-tracelet?}  In \cite{DBLP:conf/pldi/DavidY14}, $3$-tracelet, a concatenation of 3 adjacent basic-blocks, is used for searching. \textcolor{red}{Why k-tracelet is better.}


%Semantic based vulnerability detection at binary code level is an active research area and a promising direction towards securing closed-source software programs. However, we find that there is a gap in existing work \cite{pewnycross}\cite{ruttenberg2014identifying}, especially in vulnerability modelling, that is worth exploring. In the proposed vulnerability modelling techniques, it is assumed that basic-block structure is preserved across binaries \cite{pewnycross}, and thus, vulnerabilities are modelled basic-block centric. That is, semantic features are extract for from the basic-blocks in the known vulnerable function and they are compared with the basic-blocks in the target functions to identify the staring points of potential vulnerabilities.

%In \cite{ruttenberg2014identifying}, it is reported that for each basic-block in the signature (called, \textit{starting points}), first 20 basic-blocks (in-terms of similarity), in the target program, are selected for further signature matching. Similarly, in \cite{pewnycross}, for each basic-block in the signature, first 200 candidate basic-blocks, in the target program, are selected for the next level of vulnerability signature matching, using a greedy BHB (Best-Hit Boarding) algorithm. The commonality between these approaches is that the starting points, for vulnerability signature matching, are selected based on the similarity between basic-blocks in signature and target programs. However, the major drawback in this approach is that the basic-block structures can be easily distorted by factors that are beyond the control of security analysts.

%For example, figure \ref{fig:prob_stat}(a) shows the well-known heartbleed vulnerability (CVE-2014-0160) that shook the entire software industry. Figure \ref{fig:prob_stat}(a) shows the vulnerability at binary code level, compiled with GCC-4.6 and (c) shows the same vulnerability, but compiled with MinGW32. Immediately, we can say that the two programs doesn't share same basic-block structure. That is, with GCC, the vulnerable code is represented as a single basic-block, however, with MinGW, it is split into several basic-blocks. A deeper inspection would suggest that in MinGW version the library function \texttt{memcpy} is inlined while in GCC, it is not. Due to this library function inlining, in MinGW, the similarity between basic-blocks is affected considerably. That is, given the GCC version as a signature, we may miss the vulnerability in MinGW version due to basic-block \textit{splitting} and \textit{vice-versa} due to basic-block \textit{merging}. Hence, basic-block centric vulnerably modelling may not be robust against factors such as compiler type (version), optimization level and even difference in build environments. Therefore, in this paper, we present a more robust, self-adoptable vulnerability modelling technique.

%\todo {\color{blue} Do I need to include scope here? like doesn't not consider obfuscated binary that are hard to disassemble and only consider clean binaries without any obfuscations/packing.}

%\subsection{Problem Statement and Possible Solution} \label{subsec:pos_sol}
%To compare binaries, we need to define the criteria to compute similarity.
%In this work, we propose two complementary criteria to capture the semantic and structural information of the binary programs.
%Let $\mathcal{I}_1$ and $\mathcal{I}_2$ be two partial traces, we formally define the two kinds of similarity as follows:
%%The function $\langle\!\langle inst \rangle\!\rangle$ converts an instruction sequence (or an instruction) into a corresponding symbolic expression.
%
%
%\begin{mydef} \label{def:sem_sim}
%\emph{(\textbf{Semantic Similarity})
%Let $s_0$ denote a pre-state before executing $\mathcal{I}$, and $s_1=\langle\!\langle s_0, \mathcal{I}\rangle\!\rangle$ denote post-state after executing $\mathcal{I}$ on $s_0$. %post_\mathcal{I}^{pre_{\mathcal{I}}}=
%Let $S$ be set of all possible pre-states values that both $\mathcal{I}_1$ and $\mathcal{I}_2$ can execute on.
%The semantic similarity of $\mathcal{I}_1$ and $\mathcal{I}_2$, $SemSim(\mathcal{I}_1, \mathcal{I}_2)$, is defined as
%$\sum\nolimits_{s \in S} (\langle\!\langle s, \mathcal{I}_1\rangle\!\rangle -_m \langle\!\langle s, \mathcal{I}_2\rangle\!\rangle)$, where $-_m$ is a function to measure the difference of two machine states.
%%, the function $\langle\!\langle \bullet \rangle\!\rangle$ converts a partial trace into a set of \emph{symbolic expressions} and $\langle\!\langle \mathcal{I} \rangle\!\rangle(s)$ denotes assigning pre-state values $s$ to the symbolic expressions to yield the post-state values $ \mathcal{I}(s)$
%% are so \emph{similar in IO semantics} that they perform the similar computation tasks, if $PreS(\mathcal{I}_1)\approx reS(\mathcal{I}_2)$ and $PostS(\mathcal{I}_1)\approx PostS(\mathcal{I}_2)$.
%}
%\end{mydef}
%
%
%%In this work, a machine state is defined by the values stored at registers, condition-code flags and memory locations, which are common in all computing architectures.
%Semantic similarity is defined to capture the difference of the effects (i.e., post-state) of the binary program execution from the same input.
%Note that to calculate the precise effects of the program, we also need to consider the OS relevant information, like system API.
%Therefore, we consider that semantic similarity is architecture independent and OS dependent.
%
%%\ly{give one example of two code seg with different syntax, but same semantics}
%%Like example two in Fig. \ref{fig:state_seg}, the two code segments are not necessary to be from the same architecture, as the state model is not based on information that is architecture specific (see Section XX).
%
%Semantic similarity is one essential criteria to capture the behavior similarity of programs.
%However, this definition ignores the difference of program's structures, i.e., the program is treated as a black box.
%Due to the challenge \textbf{C2}, when comparing two binaries, we also want to consider the structures of the binary to make sure they are implementing the same algorithm or computation.
%%For example, bubble sort should not match with quick sort although they are semantically identical.
%This is critical in the tasks like code auditing, plagiarism detection and vulnerability detection.
%
%\begin{mydef} \label{def:comp_sim}
%\emph{(\textbf{Structural Similarity})
%The structural similarity of $\mathcal{I}_1$ and $\mathcal{I}_2$, $StrSim(\mathcal{I}_1, \mathcal{I}_2)$, is defined as
%$f^p_a(\mathcal{I}_1) -_f f^{p'}_{a'}(\mathcal{I}_2)$, where  $f^{p}_a(\mathcal{I}_1)$ (or $f^{p'}_{a'}(\mathcal{I}_2)$) is an abstraction function that maps $\mathcal{I}_1$ (or $\mathcal{I}_2$) to a structural model on architecture $a$, OS $p$ (or on architecture $a'$, OS $p'$); and $-_s$ is a function to measure the difference of two structural models.
%}
%\end{mydef}
%
%
%Structural similarity is a general definition. %, which aims to capture the high-level behavior of the binary code.
%The actual design of the abstraction function $f$ (e.g., using instruction patterns: $n$-gram~\cite{DBLP:conf/uss/JangWB13}, graphlet~\cite{DBLP:conf/raid/KrugelKMRV05} and tracelet~\cite{DBLP:conf/pldi/DavidY14}) reflects different research approaches on how to abstract the structural information of the binaries.
%Other usable structural information includes AST, PDG, data flow, type information and loop information.
%%For example, AST and PDG have been widely used for clone detection which reflect the syntax similarity.
%It is clear that the abstraction function is usually architecture and OS dependent, e.g., the codes in Fig.~\ref{fig:idiom2-ex}.
%%(e.g., using AST, PDG, data flow, type information, API and loop information)
%%Since the structureal information of assembly code (e.g., CFG) is usually architecture and OS specific, the direct matching based on CFG is not accurate, as shown in Fig. \ref{fig:prob_stat}. In this paper, we consider program information beyond the syntactic structure of the assembly code --- in addition, we also consider  data flow, type, API and loop information.
%
%%To capture the information relevant to semantics, we mainly consider data flow, API and partial traces in our model, while the structural information like BB is not the focus of our model.
%
%
%Overall, the similarity of $\mathcal{I}_1$ and $\mathcal{I}_2$ is defined as the combination of their semantic similarity and structural similarity.
%This similarity definition naturally resolves the challenge \textbf{C2}.
%
%To solve \textbf{C1}, we need to define a structural model, which is robust to different architectures and OSs. Motivated by the recent work on source code level idiom mining~\cite{allamanis2014mining},
%we propose to use the common binary patterns in different architectures and OSs as idioms, and use idioms to capture the program structural information, i.e., to define the abstraction function $f$.
%To support cross environments analysis, we propose a mapping of idioms in different architectures and OSs to a common concept such that $-_s$ can be easily computed.
%
%To solve \textbf{C3}, we propose a static analysis to capture the state-based semantics~\cite{DBLP:conf/sp/PewnyGGRH15} so that our approach is scalable.
%To capture the missing semantics related to system API using the static analysis, we introduce API idioms to complete the semantic information.
%Although the semantic comparison in this way is not precise as dynamic analysis like~\cite{egele2014blanket}, it works well with good scalability as shown in experiments.
%Furthermore, the idiom approach addresses the OS dependency problem in the semantic similarity definition.
%
%To solve \textbf{C4}, we propose a partial trace based function modeling, where partial traces of various lengths are extracted from binary code segment and they are organized in a systematic manner to handle program structure distortion. That is, by using partial trace models, no assumptions about the signature and target program structures are made (as in Tracy~\cite{DBLP:conf/pldi/DavidY14}).
%%However, in \tool, partial traces of varying lengths are considered, which doesn't make any assumption about the program structure}
%%By combining static analysis and feature hashing~\cite{jang2011bitshred}, our approach is scalable as no dynamic analysis is required.
%Overall, the unified idiom models for different environments make our solution work for all architectures and OSs.  The partial trace function modeling and complete semantic information give an accurate solution.
%
%
%%Different from \cite{allamanis2014mining}, the idiom mining is based on the tracelet from binary code and date dependency analysis. Our model is to capture the high level semantics of the assembly code, such as the idioms that abstract away the low level details, by which we search for the semantically similar assembly code regardless of architecture or OS differences,
%
%
%
%%In binary code similarity matching, due to the compilation process and optimizations, the syntax information is largely different.
%%
%% A desired model is needed to detect the assembly code in Fig \ref{fig:prob_stat} as similar, but the example in Fig \ref{fig:code_seg} not as similar. Hence, the model should be resistent to the syntactic differences due to BB structures, meanwhile, not encumbered by the issues of the pure similarly analysis of semantics. Hence, to strike a balance, we propose a partial trace (tracelet) based assembly code modeling technique.
%
%
%%To sum up, the computation model should leverages the static analysis for gaining the information from the three aspects: IO semantic based  model at the low level, tracelet based model at the middle level and idiom \cite{allamanis2014mining} based model at the high level.
%

\input{srj-overview}
\input{srj-featureCate}
%\section{Making It Accurate and Scalable} \label{sec:technicalInno}
\section{Accurate: Selective Inlining}\label{sec:inline}
%In this section, we explain the inlining process, which aims to recover the complete semantics of the binary functions.
\xyx{Inlining is a technique in compilers to optimize the binaries for maximum speed or minimum size~\cite{chang1992profile}.
During compilation, the strategy of inlining is different according to the configured optimization level, which creates the challenge for binary code search. To mitigate this issue, we propose a selective inlining strategy which is based on function invocation patterns. Note that our strategy has a different goal in contrast with compilation process --- ours is for program semantics recovery, while compilers' is for speed or size optimization.}

\subsection{Function Invocation Patterns}
\begin{figure*}[t]
   \centering
   %height=5cm,width=0.8\textwidth
  \includegraphics[width=\textwidth]{srj-figures/srj-caller-callee6.pdf}
  %\includegraphics[width=0.45\textwidth]{srj-figures/srj-caller-callee4.pdf}
  \caption{Commonly observed function invocation patterns, where `UD' denotes user-defined callee function. Here, all the incoming and outgoing edges (or calls) represent user-defined functions, unless specified as \textit{Libcall} or \textit{Termination Libcall} next to them} \label{fig:caller-callee} \vspace{-1mm}
\end{figure*}
In order to inline relevant functions, we use the function invocation patterns to guide the inlining decision. Based on our study, six commonly-observed invocation patterns are identified and summarised in Fig.~\ref{fig:caller-callee}.
Incoming (outgoing) edges in Fig.~\ref{fig:caller-callee} represent the incoming (outgoing) calls to (from) the function.
%Four out of six patterns refer to the functions to be inlined, while the other two are not.
Here, we elaborate the six patterns as follows.

\noindent\textbf{Case 1:} Fig.~\ref{fig:caller-callee}(a) depicts the direct invocation of standard C library function(s) by the caller function under investigation. To recover the semantics, it is essential to understand the semantics of called library function(s). Hence, the library function is inlined into the caller function.  Currently, we only consider the most common standard C library functions --- in total 60 functions,
%(e.g., \texttt{memcpy} and \texttt{strlen})
from both Linux (\texttt{libc}) and Windows (\texttt{msvcrt}), for inlining. This list can be further extended when necessary.

\noindent\textbf{Case 2:} Fig.~\ref{fig:caller-callee}(b) depicts the case of a recursive relationship between the caller and the UD (user-defined) callee function $f$. Hence, we inline $f$ into its caller. Note that the recursive functions are, unlike in compilers, inlined only once. For example, \texttt{gcc} has a default inlining depth of 8 for recursive functions.

\noindent\textbf{Case 3:} Fig.~\ref{fig:caller-callee}(c) depicts the common pattern of a \textit{utility function} --- e.g., the UD callee function $f$ is called by many other UD functions, while $f$ calls several \textit{library functions} and a very few (or zero) UD functions. This suggests that $f$ is behaving as a utility function, as $f$ has some semantics that is commonly needed by other functions, and hence $f$ is likely to be inlined.
% into its caller.
% depending on the number of other UD functions `referred to' by this callee (lower the reference to other UD functions, better chance of inlining).}

\noindent\textbf{Case 4:} The UD callee function $f$ in Fig.~\ref{fig:caller-callee}(d) is a variant of~\ref{fig:caller-callee}(c), where it has several references to library functions and \textit{zero} reference to other UD functions. Such zero reference to UD functions makes $f$
%carries some unique semantics --- called \textit{unique function},
an ideal candidate for inlining. Note that $f$ is inlined as the majority ($50\%$ or more) of its invoked library functions are not of termination type. Hence, we can safely assume that $f$ is doing much more than just facilitating program termination. Here, termination type refers to the library functions that lead to exceptions or program termination (e.g., \texttt{exit} and \texttt{abort}).

\noindent\textbf{Case 5:} In Fig.~\ref{fig:caller-callee}(e), The UD callee function $f$ is a variant of~\ref{fig:caller-callee}(d), which has only references to library functions.
% and \textit{zero} reference to other UD functions.
However,  all the invoked library functions in $f$ are of termination type. Hence, we consider as a function that facilitates only program termination (or exception handling) and its semantics are of little interest to the caller, which should not be inlined.

\noindent\textbf{Case 6:} Fig~\ref{fig:caller-callee}(f) depicts the scenario of a \textit{dispatcher function}  where the UD function $f$  is called by (i.e., incoming calls) many other UD functions and $f$ itself calls many other UD and library functions. In this case, $f$ appears to be a dispatcher function without much unique semantics, and hence, in most cases, not inlined.
%within the caller.

%\xyx{\textbf{Mahin, we need the example and some statistics for the above 4 cases. With the good example and solid empirical statistics, reviewers are prone to buy the following algorithm.}}


%\begin{MyAlgo}[t]{-5.5cm} %increase or decrease margin, span across columns
\begin{MyAlgo}[t]{-4.9cm} %increase or decrease margin, span across columns
\scriptsize
 \DontPrintSemicolon
 \KwData{caller $\mathcal{F}$, set of callee functions $\mathcal{C}$, set of termination lib. func. $\mathcal{L}_t$, set of inlining lib. func. $\mathcal{L}_s$}
 \KwResult{inlined function $\mathcal{F}^I$}
 \SetKwFunction{algo}{$\mathtt{SelectiveInline}$}\SetKwFunction{proc}{Extract}
 \SetKwProg{myalg}{Algorithm}{}{}
 \myalg{\algo{$\mathcal{F},\mathcal{C}, \mathcal{L}_s, \mathcal{L}_t$}}{
   %$\Re \longleftarrow \emptyset$ \;
   %$\Re_M \longleftarrow \emptyset$ \;
   %$\mathtt{dict[\cdot]=\lbrace\rbrace}$ \tcp*{n-gram dictionary}
   \ForEach{{\upshape function} f {\upshape in } $\mathcal{C}$}{
   \tcp{inline selected library functions}
   \uIf{$ f \in \mathcal{L}_s$}{
  		$ \mathcal{F}^I \longleftarrow \mathcal{F}.\mathtt{inline}(f)$\;
  		\Return $\mathcal{F}^I$ \;
	}
	\uElseIf{$f \notin \mathcal{L}_s$ { \upshape \&\& } $\mathtt{isLibCall}(f)$}{
		\Return null\;
	}
	 \tcp{for all other user-defined callee functions}
  % \tcp{below $O_u^f, O_l^f$ refers to outgoing user-defined, library func. calls and $I_u^f$ refers to incoming user-defined func. calls}		
   $I_u^f \longleftarrow \mathtt{getIncomingCalls}(f)$ \;
   $O_u^f, O_l^f \longleftarrow \mathtt{getOutgoingCalls}(f)$ \;
   $O_u^f \longleftarrow O_u^f \backslash \mathcal{F}$ \tcp*{remove recursion}	
   %\eIf{$\vert O_u^f \vert == 0$ \&\& $O_l^f \in \mathcal{L}_t $}{
   \eIf{$\vert O_u^f \vert == 0$ { \upshape \&\& } $(\vert O_l^f \cap \mathcal{L}_s\vert - \vert O_l^f \cap \mathcal{L}_t \vert )\leqslant 0 $}{
  	%$\mathtt{\textbf{return}}$ \; \label{algo2:abAPI}
  	\Return null\;
	}{
	%\tcp{measure of utility function}
	%\tcp{all other calls are considered function calls}	
  	%\xyx{$\lambda_a = \vert I_u^f \vert$ ,  	$\lambda_e = \vert O_u^f \vert$, 	$\alpha = \lambda_e/({\lambda_e + \lambda_a})$}\;
  	$\alpha = \lambda_e/({\lambda_e + \lambda_a})$ \textit{\: where} $\lambda_a = \vert I_u^f \vert$, $ \lambda_e = \vert O_u^f \vert$\;
  	%\tcp*{incoming user-defined func. calls}
   %\tcp*{outgoing user-defined func. calls}

  	%\tcp{lower the $\alpha$, $f$ is likely to be inlined into $\mathcal{F}$}
  	\tcp{lower the $\alpha$, function $f$ is likely to be inlined}
  	\eIf{$ \alpha >$ {\upshape threshold } $t$ { \upshape \&\& } $\mathtt{notRecursive}(\mathcal{F},f)$}{
  		%$\mathtt{\textbf{return}}$ \; \label{algo2:abAPI}
  		\Return null\;
		}{
		%\tcp{all other calls are considered function calls}	
  		$ \mathcal{F}^I \longleftarrow \mathcal{F}.\mathtt{inline}(f)$\;
  		\eIf{$ \vert O_u^f \vert > 0$}{
  		$\mathtt{SelectiveInline}(f,O_u^f,\mathcal{L}_s, \mathcal{L}_t)$\; \label{algo2:abAPI}
		}{
		\Return $\mathcal{F}^I$ \;
		}
 		}
  	}
}
\Return $\mathcal{F}^I$ \;
}
 \caption{Selective inlining algorithm}\label{algo:select-inline}
\end{MyAlgo}

\subsection{Inline Decision Algorithm}

From the discussions above, in Fig.~\ref{fig:caller-callee} (a), (b), (d) and (e) there is a clear criterion in deciding whether a callee
%\xyx{(no matter user-defined or library function)}
should be inlined or not.
%For \ref{fig:caller-callee}(c) (Case 4), we inline the callee, if the total number of termination type library functions are smaller than interesting library functions .
However, for Fig.~\ref{fig:caller-callee} (c) and (f), the most commonly observed invocation patterns, a systematic decision-making procedure is still needed.
To identify the cases of commonly-used functions (i.e., utility functions) to be inlined, we borrow the \textit{coupling} concept from the software quality and architecture recovery community. In software metrics, the coupling between two software packages is measured by the dependency between them, e.g., the software package instability metrics~\cite{martin2003agile}. In this work, similarly, we measure the function coupling by the \textit{function coupling score}. % to decide whether a callee should be inlined or not, which is formally defined as follows.

\begin{mydef} \label{def:comp_semantic}
\emph{\textbf{Function Coupling Score}}
refers to the complexity of the invocations involved in a function and is calculated as $\alpha = \lambda_e/({\lambda_e + \lambda_a})$,
%\begin{equation}
%\begin{aligned}
% \alpha = \frac{\lambda_e}{\lambda_e + \lambda_a}
%\end{aligned}
%\end{equation}
where $\lambda_a$ represents the number of UD functions that refers to the callee, and  $\lambda_e$ represents the number of UD functions that is referred to by the callee.
\end{mydef}
The lower the value of $\alpha$, more likely the callee should be inlined. The rationale is that the low function coupling score implies the high independency of the functionality ---  the high possibility of being utility function.
%For example, when the callee doesn't refer to any other UD functions (i.e., $\lambda_e = 0$), it is assumed to be behaving as autility function (where, $\alpha = 0$) and hence, inlined.
When calculating $\lambda_e$, we only consider the UD functions invoked by the callee, not the library functions. As mentioned in case 3 (Fig.~\ref{fig:caller-callee}(c)) and 6 (Fig.~\ref{fig:caller-callee}(f)), it is the invocations of UD functions that indicate the behavior of the callee as a dispatcher or a utility function, not the invocations of library functions.

%\xyx{\textbf{Mahin, need to give $\alpha$  for (a) to (b).}} \mahin{yes, I'll include after exp.}

The proposed selective inlining process is presented in Algorithm~\ref{algo:select-inline}. As first, if the callee is one of the selected library functions (e.g., \texttt{memcpy} and \texttt{strlen} as in case 1), it is directly inlined into the caller (lines 3-5) and rest of the library functions are ignored. Next, the UD functions that refer to the callee is denoted as $I_u^f$; while the library and UD functions  referred to by the callee are identified and denoted as $O_l^f$ and $O_u^f$, respectively at line 8-9. Callee that invokes only library functions that are of termination type is not inlined (lines 11-12). Finally, for the rest of the callee functions, the function coupling score is calculated and if it is  below some threshold value $t$, the callee is inlined (lines 13-24). This recursive procedure is continued until all the related functions are analysed.
%\textbf{Mahin, how to distinguish b, c and d is not clear.}

In our preliminary study on BusyBox compiled for x86 32bit, we identify that 14 UD utility functions (case 3) have more than 50 incoming calls with 1 outgoing call, while 12 UD dispatcher functions (case 6) have more than 50 outgoing calls with just 1 incoming call.  Similarly, in BusyBox compiled for ARM, we identify such 15 UD utility functions but only 4 UD dispatcher functions. This clearly differentiates the function invocation patterns case 3 (Fig.~\ref{fig:caller-callee}(c)) and 6 (Fig.~\ref{fig:caller-callee}(f)).
In this work, we adopt a lightweight static analysis to make the inlining decision for the performance reason (as shown in Section~\ref{sec:experiemntation}).
A more expensive program analysis can be used to improve the accuracy, but we strike for the balance between performance and accuracy in this work.



 %The rationale is that the emulation step is after the selective inlining step, and the important function calls have been retained. For those function calls after selective inlining,  these functions as well as other functions they called will be all inlined.

%srj
%aum gaanathipathaye namaha
 %aum gaanathipathaye namaha
%srj


\section{Scalable Function Matching}\label{sec:func_match}
%This section, we present the four modules in \tool in details.
%As the first step, \tool disassembles the binary and splits it into basic functional units or functions. Then, it constructs the control-flow graph (CFG) for each function, where each CFG is consists of basic-blocks. Later, the constructed CFGs are used to generate the partial traces.

%\subsection{Partial Trace Extraction} \label{subsec:partial_trace}

In \tool and \toolNew, similarity matching is done at the granularity of function and sub-function levels (i.e., we can match a part of a target function to the signature function).




\subsection{Primary Matching} \label{subsec:matching:primary}
In primary matching, high-level semantic features and structural features are combined to measure the semantic similarity of the two code segments as the granularity of function. Let $\mathcal{SIM_{H}}(sig, tar)$  denote the similarity score of using high-level semantic features and $\mathcal{SIM_{S}}(sig, tar)$ denote the similarity score of using structural features.
The former score is measured by Jaccard containment similarity~\cite{agrawal2010indexing}, considering each function has a bag of high-level semantic features:
\begin{equation}
\begin{aligned}
\mathcal{SIM_{H}}(sig, tar) = \frac{\mathcal{H}_{sig} \bigcap \mathcal{H}_{tar}}{\mathcal{H}_{sig}}
\end{aligned}
\end{equation}

\xyx{Specifically, for each of six types of high-level semantic features, the Jaccard distance is calculated, and the overall $\mathcal{SIM_{H}}(sig, tar)$  is the mean value of Jaccard distances of different types of high-level semantic features. Note that if both signature function and target function have no function calls, then features in terms of function call tags, function call sequence, function parameters are not available, and they are not counted in overall similarity.}

The latter score is measured by calculating FDD (Definition 3 in \S\ref{sec:category:structralFea}), and the FDD value is already normalized into  the range between 0 to 1. $\mathcal{SIM_{S}}(sig, tar)$ is to convert the distance to the similarity:
\begin{equation}
\begin{aligned}
 \mathcal{SIM_{S}}(sig, tar) = Convert(FDD(sig, tar)) \\
 = 1 - FDD(sig, tar)%\frac{FDD(sig, tar)}{w'_{sig} + w'_{tar}}
\end{aligned}
\end{equation}
%\begin{equation}
%\begin{aligned}
% Norm(FDD(sig, tar))
%\end{aligned}
%\end{equation}
%where $w'_{sig}$ and $w'_{tar}$ denote the number of weighted assembly instructions in function $sig$ and function $tar$, respectively.
The idea of convertion is that if these two functions' distance is closer to 0, their similarity is closer to 1. Vice versa, if their distance is as big as it could possibly be, their similarity is prone to be 0.

Once we have similarity scores in high-level semantics and structures, the primary matching result $\mathcal{SIM^\prime}(sig, tar)$ is calculated as the weighted sum of $\mathcal{SIM_{H}}(sig, tar)$ and $\mathcal{SIM_{S}}(sig, tar)$, which is defined as follow:
\begin{equation}
\begin{aligned}
 \mathcal{SIM^\prime}(sig, tar) =  (1 - \mathcal{W}/2) *  \mathcal{SIM_{H}}(sig, tar)  \\
  + ( \mathcal{W}/2) * \mathcal{SIM_{S}}(sig, tar)
\end{aligned}
\end{equation}
where $\mathcal{W}$ = $min(\frac{w'_{sig}}{ w'_{tar}}, \frac{w'_{tar}}{ w'_{sig}})$ and $\mathcal{W}$ means the ratio of these two functions' instruction number.

Hence, the idea of primary matching is straightforward --- structural features are effective when the sizes of two functions are similar. If two functions have the similar size of instructions, structural features have the same weight as high-level semantic features. If not, high-level semantic features have more weight than structural features. According to this idea, a high value of $\mathcal{SIM^\prime}(sig, tar)$ (e.g., $\geq 0.8$) indicates that $sig$ and $tar$ are similar in both structures and high-level semantics.

\subsection{Low-level Semantic Features Matching} \label{subsec:matching:lowFea}
%\footnote{Signature function refers to the search query function in hand, whereas target functions refer to functions in the target binary pool against which the signature function is matched.}
When the primary matching does not find any ranked results that are of high $\mathcal{SIM^\prime}(sig, tar)$ values, low-level semantic features are used.
Note that low-level semantics refer to not only the implementation details, but also the partial semantics of the function.
To this end, we propose the function model consisting of partial traces of various lengths, %, a variant of tracelet~\cite{DBLP:conf/pldi/DavidY14},
which is more flexible in terms of comparison granularity, compared with the BB-centric
%and tracelet~\cite{DBLP:conf/pldi/DavidY14} based
function modelling~\cite{DBLP:conf/sp/PewnyGGRH15,luo2014semantics}. Then, from these partial traces we extract symbolic expressions. Based on I/O values for register, flags and memory addresses, we match the partial traces inside two functions. %To reduce the noise, we remove the partial traces that are infeasible to reach (via solving the symbolic expressions) or are specific to compilers.
Last, we apply Jaccard containment similarity~\cite{agrawal2010indexing} to measure the similarity score of two function models.
%That is, for each function, we generate partial traces of various lengths (called, \textit{k-length partial traces}), where by varying the length, the semantics of the function is captured at various granularity levels.
%granularity of single building block is adjusted\footnote{Length of one yields a basic-block centric models, where the basic-block is considered as a single building block.}.
\subsubsection{K-length Partial Trace Extraction} \label{subsec:partial_trace_ext}
Our partial trace extraction is based on the technique proposed by David \emph{et. al.}~\cite{DBLP:conf/pldi/DavidY14}.
We omit the algorithm and explain the results using one example.
Fig.~\ref{fig:example-cfg} depicts a sample CFG of a function and the extracted 2- and 3-length partial traces (i.e., for $k=2,3$). We can observe that the original control-flow instructions (\texttt{jnb}, and \texttt{jb}) are omitted as the flow of execution is already determined. Note that the feasibility of the flow of executions is not considered at this step. %In \S\ref{subsubsec:sou_prun}, we show how the partial traces which are \textit{infeasible} to reach are identified via constraint solving and removed. %from our analysis. Here, infeasible partial traces refer to program execution flows that are not possible to reach during any concrete execution.

\begin{figure}[]
\begin{center}\vspace{-3mm}
\includegraphics[width=7.5cm]{srj-figures/srj-partial_trace_ex.pdf} %\vspace{-1mm}
\caption{A sample CFG (a) and the extracted 3-length (b) and 2-length (c) partial traces}
\label{fig:example-cfg} \vspace{-2mm}
\end{center}
\end{figure}
%
%\subsection{Trace Pruning}\label{subsec:trace_prun}
%In \tool, we adopt two trace pruning methods to reduce the noise in the results and also to speed up the matching process.
%%steps to reduce the noise in I/O samples based function \textit{similarity} matching, which is less precise than the unscalable function \textit{equivalence} matching.}
%\subsubsection{Infeasible Partial Trace Pruning} \label{subsubsec:sou_prun}
%\tool is a static analysis based tool, and therefore it is difficult (or even impossible) to identify all the infeasible partial traces in practice. In \tool, we prune the obviously infeasible partial traces via the constraint solver. Given the symbolic expressions extracted from a partial trace, we rely on the constraint solver to determine whether all the constraints present in the symbolic expressions can be satisfied. %If not, that relevant partial trace is considered infeasible.
%As we use the constraint solver to generate I/O samples in \S\ref{subsec:sem_fea_ext}, we need no additional effort to identify the infeasible partial traces --- if the constraint solver is unable to find appropriate concrete values for the pre- and post-state variables, the relevant partial trace is considered infeasible.
%%That is, given a partial trace, we extract the set of I/O formulas, as explained in Section~\ref{subsec:stat_sem}, and try to generate a model that satisfied all the constraints present in the I/O formulas (i.e., able to generate appropriate concrete values for input/out vocabularies), if such model is not available then that partial trace is considered infeasible.
%%and pass them to the solver, which will try to solve the constraints and generate a model that satisfies them all. If the solver fails to generate a model, we label that partial trace as infeasible and prune it from the analysis.
%
%We also observe that some  partial traces, for which the solver is able to generate models (or I/O samples), might be infeasible during actual execution, as the feasibility of their paths depends on various factors, such as global variables, values in the heap and other dynamic data, that are beyond the scope of our analysis. Further, in general, infeasible path elimination (or detection) by itself is a hard problem~\cite{bodik1997refining}. However, compared with the static analysis solutions proposed in the literature~\cite{DBLP:conf/pldi/DavidY14,pewny2014leveraging,DBLP:conf/sp/PewnyGGRH15}, this work makes an attempt to reduce the search candidates using a pruning technique.
%
%%Even with the infeasible partial traces pruned, it is still difficult to compare the signature function against those functions in the target binary pool. Thus, a systematic way of performing the similarity matching among partial traces is desired. Specifically, the question ``what length of partial traces in the search query needs to be matched with what length of partial traces in the target function?'' still persists.
%
%
%
%
%\subsubsection{Compiler Specific Code Pruning}
%%%Compiler idioms capture the compilation information, via code patterns  commonly  included for a specific compilation option during compilation process.
%%%\noindent \textbf{Compiler idiom}
%%%These idioms help to identify the additional code included by the compiler during compilation process.
%Based on the compilation option selected, the compiler could include additional code (i.e., code that is not originally written by the programmer) into the compiled binary. For example, there are more than 150 compiler options available for both \texttt{gcc} and \texttt{MS Visual Studio}, which a programmer can choose from when compiling her program. Some of these options result in adding extra code into the binary (e.g., \texttt{stack-smashing protection} or SSP shown in Fig.~\ref{fig:gcc_ssp}). The code segments in the function prologue and epilogue in Fig.~\ref{fig:gcc_ssp} ensure the stack integrity is not violated. These code segments are automatically included by the \texttt{gcc} compiler when the option for stack smash protection is enabled.
%%(using either \texttt{-fstack-protector-all} or \texttt{-fstack-protector} flag).
%%(using either -$\mathtt{fstack}$-$\mathtt{protector}$-$\mathtt{all}$ or -$\mathtt{fstack}$-$\mathtt{protector}$ flag).
%
%In similarity matching between the signature and target functions, the additional compiler specific code can introduce noise by changing the code structures and diluting the similar parts, especially when the functions are small.
% %the additional code included by the compiler might influence the similarity score between the signature and target functions, especially when the functions are relatively small.
%% Therefore, in order to avoid imprecise similarity matching, the compiler specific code need to be removed from the partial traces.
%Thus, it is necessary to identify and remove the compiler specific code from the partial traces.
%However, directly removing some code from a partial trace might lead to incorrect semantic features, as these features are very sensitive to the underlying code semantics.
%
%To this end, we propose a conservative approach to address this problem by generalizing the compiler specific code into some patterns and systematically pruning the partial traces that contain these patterns.
%%, as in~\cite{rosenblum2008learning},
%%The partial traces that contain these patterns are systematically pruned.
%That is, instead of removing the compiler specific code from a partial trace, which is error prone, we just remove the partial trace itself if the compiler specific code pattern accounts for the majority (50\% or more) of the code. As an initial step, we only consider three types of compiler specific code patterns (i.e., SSP, function prologue and epilogue) that are very commonly observed in the binaries. This list can be further extended when necessary.
%of the code present in a partial trace, the partial trace is completely removed from the analysis.
%This approach does not eliminate the whole the problem but tries to minimize the effect of compiler specific code influencing the final similarity score.



%Interestingly, some compiler idioms help to understand more about the program. For example, the presence of SSP in selected functions (i.e., -$\mathtt{fstack}$-$\mathtt{protector}$ compiler feature) in a program indicates that these  functions have buffers larger than 8 bytes. Compiler idioms are architecture- and OS-specific, however, their high-level meanings are unified across architectures and OSs.


\subsubsection{Function Model Generation} \label{subsec:fun_mod_mat}

\begin{figure}[t]
  \centering
  %width = 7cm,
  \includegraphics[width = 5cm,height=4cm]{srj-figures/srj-func_mod.pdf} %the_graph
  \caption{A sample signature (a) and target (b) functions, where the lines indicate the matched partial traces and the nodes refer to the basic-blocks.} \label{fig:func_mod}
%\vspace{-1mm}
\end{figure}


In \tool, we leverage on length variant partial traces to model the functions. %The key advantage of using partial trace over other techniques~\cite{DBLP:conf/pldi/DavidY14, DBLP:conf/sp/PewnyGGRH15,luo2014semantics,sebastian2016discovre} lies in its resilience to the structural changes. %To this end,
We generate partial traces with three different lengths (i.e., $k=1,2$ and $3$) from each function, all of which collectively constitute the function model. For example, function models generated from the signature ($\mathcal{M}_{sig}$) and target ($\mathcal{M}_{tar}$) functions in Fig.~\ref{fig:func_mod} are listed as follows:

\begin{itemize}
\small
\itemsep0em
  \item[] $\mathtt{\mathcal{M}_{sig}:  \lbrace \langle 1 \rangle,  \langle 2 \rangle,\ldots, \langle 1,2 \rangle, \langle 2,5 \rangle, \ldots, \langle 1,2,4 \rangle, \langle 2,4,7 \rangle,\ldots\rbrace}$
   \item[] $\mathtt{\mathcal{M}_{tar}: \lbrace \langle c \rangle,  \langle b \rangle, \ldots, \langle a,b \rangle, \langle b,c \rangle, \ldots, \langle a,b,c \rangle, \langle b,c,f \rangle, \ldots \rbrace}$
\end{itemize}

In \tool and \toolNew, we support the function model similarity matching in terms of \textit{n-to-m}, \textit{1-to-n}, \textit{n-to-1} and \textit{1-to-1} partial trace matching across the signature and target functions, mitigating the impact of program structural difference.
\begin{description}
\itemsep0em
  \item[\textit{n-to-m}] Partial traces of length $n (\in\mathbb Z_{> 1})$ generated from the signature function are matched against the partial traces of length $m (\in\mathbb Z_{> 1})$ generated from target function. In Fig.~\ref{fig:func_mod}, matching partial traces $\langle 3,6,8\rangle$ and $\langle d,g\rangle$ is \textit{3-to-2} matching.
  \item[\textit{1-to-n}] A basic-block (i.e., a partial trace of length 1) in the signature function is matched against the partial traces of   length $n (\in\mathbb Z_{> 1})$ generated from target function. In Fig.~\ref{fig:func_mod}, partial traces $\langle 1\rangle$ and $\langle 5\rangle$ are matched with $\langle a,b\rangle$ and $\langle f,h,i\rangle$, respectively.
  \item[\textit{n-to-1}] Partial traces of length $n (\in\mathbb Z_{> 1})$ generated from the signature function are matched against a basic-block in the target function. In Fig.~\ref{fig:func_mod}, partial trace $\langle 4,7,9 \rangle$ is matched with $\langle e \rangle$.
  \item[\textit{1-to-1}] A basic-block in the signature function is matched against a basic-block in the target function. It is also known as pairwise comparison at the level of single basic-block. In Fig.~\ref{fig:func_mod}, partial trace $\langle 2\rangle$ is matched with $\langle c\rangle$.
\end{description}

\textit{1-to-n} matching addresses the issue of basic-block splitting --- a single basic block in the signature function is split into several smaller basic blocks in the target function. Similarly, \textit{n-to-1} matching addresses the basic-block merging problem.
%On the other hand, \textit{n-to-m} mapping is generally preferred when the signature (target) function is part of a huge target (signature) function and appropriately selected values for $n$ and $m$ will maximize the function similarity. Finally, if none of the aforementioned matching techniques work, we resort to \textit{1-to-1} matching to compare the basic-block similarity in a pairwise way, which is similar to those basic-block centric comparison approaches~\cite{DBLP:conf/sp/PewnyGGRH15,luo2014semantics}.
In tracelet modeling \cite{DBLP:conf/pldi/DavidY14}, authors recommended that both the signature and target should be of the same size (i.e., $k=3$), and hence only \textit{n-to-n} matching is performed.
In~\cite{DBLP:conf/sp/PewnyGGRH15} and~\cite{luo2014semantics}, pairwise comparison of single basic-block (i.e., \textit{1-to-1} matching) is performed as an initial step to shortlist target functions. In contrast, in \tool and \toolNew, all the  4 types of function matching are needed to cover all possible BB-structure variances that arise due to architecture, OS and compiler differences.


For the similarity score of using low-level semantic features $\mathcal{SIM_{L}}(sig, tar)$, considering the function model as a bag of partial traces, Jaccard containment similarity~\cite{agrawal2010indexing} is also used to measure the similarity of two different function models:
\begin{equation}
\begin{aligned}
 \mathcal{SIM_{L}}(sig, tar)  = \frac{\mathcal{M}_{sig} \bigcap \mathcal{M}_{tar}}{\mathcal{M}_{sig}}
\end{aligned}
\end{equation}
where $\mathcal{M}_{sig}$ and $\mathcal{M}_{tar}$ refer to function models of low-level semantics that are generated from signature and target functions, respectively.

%
%Finally, considering the function model as a bag of partiral traces, Jaccard containment similarity~\cite{agrawal2010indexing} is used to measure the similarity score between  two different function models, and is defined below:
%\begin{equation}
%\begin{aligned}
% sim(\mathcal{M}_{sig}, \mathcal{M}_{tar}) = \frac{\mathcal{M}_{sig} \bigcap \mathcal{M}_{tar}}{\mathcal{M}_{sig}}
%\end{aligned}
%\end{equation}
%where $\mathcal{M}_{sig}$ and $\mathcal{M}_{tar}$ refer to function models generated from signature and target functions, respectively.

%In \tool, the function model matching is not fixed, where, for a given signature function, the searching algorithm will try all possible function models and pick the best one that maximizes the signature-target function similarity. In contrast, the techniques proposed in the literature do not have the flexibility to try out all possible matchings. For example, in tracelet-based modeling \cite{DBLP:conf/pldi/DavidY14}, the authors recommended that the tracelet size should be larger (e.g., $k>2$), for both signature and target functions, hence, only \textit{n-to-m} matching is performed, in fact, it is \textit{n-to-n} matching as they consider the same tracelet size for both signature and target functions. Further, in~\cite{DBLP:conf/sp/PewnyGGRH15} and~\cite{luo2014semantics}, pairwise comparison of basic-blocks (i.e., \textit{1-to-1} matching) is performed as an initial step to identify potential target functions, which inherently assumes that the program structure is maintained across signature and target functions and at least one basic-block in the target function resembles a basic-block in the signature function.

%\subsection{Locality Sensitive Hashing} \label{subsec:lsh} %\note{replace this with simple feature hasing technique, if no time, leave as it is.. not very important}
%In \tool, we leverage on Locality-sensitive Hashing (LSH) technique to perform the function matching more efficiently and in a scalable manner. Locality-sensitive hashing is an algorithm for searching similar items in large and high dimensional dataset \cite{leskovec2014mining} based on the assumption that, if two items are similar, the hashed value of the two items will remain similar. In \tool, we use MinHashing~\cite{broder1997resemblance} technique to hash the extracted features and generate hash signature for each function model. To this end, we use 1000 (i.e., $n=1000$) hash functions to generate the signature, which leads to an error of 3.16\%. The similarity between two function model is given by this equation:
%\begin{equation}
%\begin{aligned}
% sim(mh_a, mh_b) = \frac{\vert mh_a[i]=mh_b[i ]\vert}{n}
%\end{aligned}
%\end{equation}
%where the jaccard similarity between two function models is approximated by MinHash similarity between the two models.

%\subsubsection{Function Model Generation} \label{subsec:fun_mod_mat}

%\subsection{Primary Matching and Final Matching} \label{subsec:sem_fea_ext}
%\subsubsection{Primary Matching}
\subsection{Final Matching} \label{subsec:matching:final}

To remove false positives of using low-level semantics alone, $\mathcal{SIM^\prime}(sig, tar)$ is also considered in the final matching.

Let $\mathcal{SIM^*}(sig, tar)$ denote the overall similarity score calculated in the final matching. $\mathcal{SIM^*}(sig, tar)$ is calculated as the weighted sum of $\mathcal{SIM^\prime}(sig, tar)$ and  $\mathcal{SIM_{L}}(sig, tar)$. Here, we adopt two strategies for different matching scenarios.

If the matching is based on the same code base, we consider both $\mathcal{SIM_{L}}(sig, tar)$ and $\mathcal{SIM^\prime}(sig, tar)$ are equally important.
\begin{equation}
\begin{aligned}
 \mathcal{SIM^*}(sig, tar) =  (1/2) * \mathcal{SIM^\prime}(sig, tar)  \\
  + (1/2) * \mathcal{SIM_{L}}(sig, tar)
\end{aligned}
\end{equation}


Otherwise, if the matching is based on different code bases (i.e., experiments on cross-OS matching), we ignore the structural features and give equal weight for low-level semantics and high-level semantics.
\begin{equation}
\begin{aligned}
 \mathcal{SIM^*}(sig, tar) =  (1/2) * \mathcal{SIM_H}(sig, tar)  \\
  + (1/2) * \mathcal{SIM_{L}}(sig, tar)
\end{aligned}
\end{equation}

\xyx{Still take the code segments in Fig.~\ref{fig:falseposi} as example. The two code segments are from different code bases, and hence the BB structure will not be similar. Low-level semantics and high-level semantics are used together to measure their semantic similarity. Previously,  in \tool, using low-level semantics alone suggests they have a similarly value of 0.7 --- 70\% of tracelets in the signature function match the target function on I/O value pairs. Now, in \toolNew, since they have a low high-level similarity of 0.7 and no similarity in high-level features and structural features, their overall similarity is lower to 0.35. Hence, the false positive has a lower similarity score, and will be removed from the best-match or top-5-match list.}


%srj
%aum gaanathipathaye namaha


%
%\begin{figure}[t]
%\begin{center}%\vspace{-1mm}
%\includegraphics[width=0.4\textwidth]{srj-figures/srj-abs-1.pdf}
%%\vspace{-1mm}
%\caption{Library function abstraction levels}
%\label{fig:abs} \vspace{-2mm}
%\end{center}
%\end{figure}
%
%

%\subsection{Emulation}\label{sec:scalable:emulation}
%\subsection{Function Filtering}\label{sec:prefilter}

%%In practise, %to effectively search for a semantically similar function from a pool of several hundred thousand (or more) target functions compiled for various architectures and OS using different types of compilers with varying optimization levels,
%To deal with a huge number of target functions in real-world binaries, an effective filtering process can remove irrelevant target functions before the expensive matching step.
% \tool leverages on three types of filters, starting from the specific one to the most general one, to shortlist the candidate target functions. \\
%
%\noindent \textbf{Filter 1}: The first type of filters %\xyx{(\textit{aka.,} precise)}
% looks for identical library call invocations in the target functions according to the signature function. If the identical invocations are found, the corresponding targets functions are good candidates for further matching. The reason is that the library call invocations provide an important partial semantics of the function.
%However, this filter is OS dependent and fails to support library calls that have different names yet with similar functionality (e.g., \texttt{memcpy} and \texttt{memmove}).
%%Library calls can be inlined by the compiler or programmers might implement the same functionality in user-defined functions. Thus, relying on library call names will fail. %In addition, the given function may not invoke any library call at all, in such situations, library call based filtering is not helpful.
%
%\noindent \textbf{Filter 2}: To address the problem in library call name matching, we consider the library call operation types, which help to match the high-level functionality of the library calls. As shown in Fig.~\ref{fig:abs}, there are several ways of abstracting the functionality of a library call. Abstraction level 0 gives the base operation type (e.g., \textit{string} operation), while level 1 gives a more concrete but general abstraction supported cross all OSs.
%For example, \texttt{strcpy} and \texttt{strcat} can be mapped to \textit{string manipulation} operation. Therefore, we adopt abstraction level 1 to summarize the library function behavior. In addition, filter 2 is OS neutral, where functions that perform similar task but with different names across OS are mapped to the same operation type, e.g., \texttt{malloc} (Windows/Linux) and \texttt{HeapAlloc} (Windows) are mapped to \emph{memory allocation} operation type as shown in Fig.~\ref{fig:abs}.
%%\ly{I want to highlight this filter 2 is OS neatural and use one example to show it in fig 4. Also we should say we maintain the mapping ourself.}
%%in this study.
%%\todo{Include example for op. type mapping}
%
%However, library calls can be inlined by the compiler or programmers might implement the same functionality in user-defined functions. Thus, relying on library call names or the operation type will fail. To address this, we propose filter 3.%In addition, the given function may not invoke any library call at all, in such situations, library call based filtering is not helpful.
%
%
%\noindent \textbf{Filter 3}: %To address these serious issues of the first two filters, we \xyx{propose}
%The third filter is designed look for similarities in the instruction types involved in a binary function as a whole. Here, instruction type refers to a high-level operation carried out by an instruction~\cite{kruegel2005polymorphic}. In total, instructions are categorized into 14 and 8 instruction types for Intel and ARM architectures, respectively. For example, \texttt{mov} instruction is mapped to \textit{data movement} instruction type while \texttt{push} is mapped to \textit{stack operation}.
%In addition, instruction types also support cross-architecture function matching, where assembly instructions from ARM and Intel architectures are mapped to the same instruction type even though they are not identical at the instruction level (e.g., \texttt{call} (Intel) and \texttt{bl} (ARM) can be mapped to \textit{invoke function} instruction type).
%
%Nevertheless, this filter may suffer from the problem of semantics matching --- the instruction types used to implement the functions might look very similar even though they have totally different functionalities at a higher level.
%Since the filtering process is to shortlist all the target functions that are similar to the signature. %, and hence similarity at the instruction level is also very relevant even though they might be different at the high-level function behaviour.
%%This conservative approach is adapted not to miss any potential target functions.
%For the propose of not missing any potential target functions, some semantically irrelevant functions are tolerated in this step. \\
%
%
%\noindent \textbf{Filtering Algorithm} Filter 1 is specific and OS dependent. Filter 2 and 3 are general and cross-OS and cross-architecture.
%Our filtering process is shown in Algorithm~\ref{algo:pre-filt}.
%At lines 7,9 and 11, we use Jaccard distance~\cite{cha2007comprehensive} to measure the similarity between the signature and target functions in terms of each  filter, i.e., their similarity in identical library calls, in library call operation types, and in instruction types.
%Following the design of applying the filters one by one from the most specific one to the general one, we set the weights ($w_1 > w_2 > w_3 >0$ ) to the similarities achieved by Filter 1 to 3. At line 12, we sort the candidate functions according to the overall similarity on three filters (calculated at line 10). Finally, at line 13, we get the top $N$ of the sorting results and use them for function model matching.
%%\xyx{\textbf{give concrete value in implementation, discuss about the weight set in threat to validity!}}% where filter 1 is the most precise with  $w_1=1.0$, and filter 3 is the least precise with $w_3=0.5$ while filter 2 being in the middle with $w_2=0.8$.
%Note that our filtering process is performed after selective inlining step, hence we keep a mapping from the candidate function to its invoked libraries in order to apply the filters.
%%\todo{mention how NDSS 2016 gonna fail in pre-filtering}, \xyx{Mahin, maybe we can put all the comparison with NDSS paper in related work.}
%
%%sIn addition, we use inlined target functions, which helps to
%
%%It is important to note that, for each library call, we obtain the corresponding instruction types from the actual implementation of that library function in \texttt{libc} and \texttt{msvcrt}, for Linux and Windows binaries, respectively.
%
%
%%To make \tool capable of analyzing cross-OS binaries, we use a multi-level abstraction function $\mathtt{abstractSystemAPI}$ at line \ref{algo2:abAPI} to abstract system APIs based on their type. For example, the system API (or library call) \texttt{strlen} deals with string objects, and hence  the API can be naturally abstracted to `string' type. To this end, we use two levels of granularity (i.e., level 0 and 1) to abstract the system APIs, where each abstraction level play a different role in \tool~ --- abstraction level 0 abstracts the system API to their basic type, whereas level 1 provides more meaningful information about the API. For example, using our abstraction function, the system APIs \texttt{strlen}, \texttt{strcpy} and \texttt{strncpy} can be abstracted into two levels shown in Fig.~\ref{fig:abs}.
%
%
%%Based on the requirement of the analyst, there may be several ways to abstract a system API. As shown in Fig.~\ref{fig:abs}(b), abstraction level 1 can be more expressive in providing more meaningful information about the API or it can be limited as in Fig.~\ref{fig:abs}(a). One of the key applications of abstraction level 0 is that it is mostly used in pre-filtering process, where if a signature involves string manipulation operations then it is wise to quickly retrieve the target programs that also involve string manipulations. Similarly, abstraction level 1 is used to precisely match the signature with the target programs, where it can be used to specify additional constraints for the matching process. Level 1 abstraction is quite useful for vulnerability signature matching, e.g., the analyst can remove functions, from the filtered target programs, that use secure system API (e.g., \texttt{strncpy}, \texttt{\_\_strncpy\_chk,}etc.\footnote{With \texttt{FORTIFY\_SOURCE} compiler feature, whenever possible, \texttt{gcc} tries to uses buffer-length aware replacements for functions like \texttt{strcpy}, \texttt{memcpy}, \texttt{memset}, \texttt{gets}, etc., which are more secure.}), which are less likely to contain an exploitable vulnerability.
%%\xyx{Our evaluation also showed that ...}
%%
%%\subsection{Cross-architecture, OS and compiler pre-filtering algorithm}
%%To this end, in \tool, we have proposed a across-architecture, OS and compiler friendly pre-filtering algorithm that can filer the candidate target functions in a scalable fashion.
%
%%
%%\begin{MyAlgo}[t]{-4.8cm} %increase or decrease margin, span across columns
%%\scriptsize
%% \DontPrintSemicolon
%% \KwData{signature function $f$, set of target functions $\mathcal{T}$}
%% \KwResult{set of candidate target functions $\mathcal{T}_c$}
%% \SetKwFunction{algo}{$\mathtt{FunFilter}$}\SetKwFunction{proc}{Extract}
%% \SetKwProg{myalg}{Algorithm}{}{}
%% \myalg{\algo{}}{
%%    %$f_{in} \longleftarrow \mathtt{getInlinedFunc}(f)$\;
%% 	$\mathcal{S} \longleftarrow \lbrace \rbrace$ \;
%% 	%$t_{in} \longleftarrow \mathtt{getInlinedFunc}(t)$\;
%%   \ForEach{{\upshape function} t {\upshape in } $\mathcal{T}$}{
%%  	$s^t \longleftarrow \emptyset$\; 	
%%   $\mathcal{L}_f \longleftarrow \mathtt{getLibFuncList}(f)$ \;
%%   %$t_{in} \longleftarrow \mathtt{getInlinedFunc}(t)$\;
%%   %\uIf{$\vert \mathcal{L}_f \vert > 0)$}{
%%   \ForEach{{\upshape libcall} l {\upshape in } $\mathcal{L}_f$}{
%%   	$s^t \pluseq w_1 * \mathtt{getLibFuncNameSim}(l,t)$ \;
%%   	$l^O \longleftarrow \mathtt{getOpType}(l)$\; 	
%%   	$s^t \pluseq w_2 * \mathtt{getLibFuncOpTypeSim}(l^O,t)$ \;
%%   	%$l^I \longleftarrow \mathtt{getInstType}(l)$\; 		
%%   	%$s^t \pluseq w_3 * \mathtt{getLibFuncInstrTypeSim}(l^I,t)$ \;
%%   }
%%   $s^t \pluseq w_3 * \mathtt{getFuncInstrTypeSim}(f,t)$ \;
%%   $S[t] = s^t$ \;
%%}
%%$\mathcal{S}^s \longleftarrow \mathtt{sortCanditTargetFunc}(S)$ \;
%%$\mathcal{T}_c \longleftarrow \mathtt{topNTargetFunc}(\mathcal{S}^s,\mathcal{N})$ \;
%%\Return ${\mathcal{T}_c}$ \;
%%}
%% \caption{Function Filtering Algorithm}\label{algo:pre-filt}
%%\end{MyAlgo}
%
%
%\begin{MyAlgo}[t]{-4.8cm} %increase or decrease margin, span across columns
%\scriptsize
% \DontPrintSemicolon
% \KwData{signature function $f$, set of target functions $\mathcal{T}$}
% \KwResult{set of candidate target functions $\mathcal{T}_c$}
% \SetKwFunction{algo}{$\mathtt{FunFilter}$}\SetKwFunction{proc}{Extract}
% \SetKwProg{myalg}{Algorithm}{}{}
% \myalg{\algo{}}{
%    %$f_{in} \longleftarrow \mathtt{getInlinedFunc}(f)$\;
% 	$\mathcal{S} \longleftarrow \lbrace \rbrace$ \tcp*{store similarity score}
% 	%$t_{in} \longleftarrow \mathtt{getInlinedFunc}(t)$\;
% 	%$\mathcal{L}_f \longleftarrow \mathtt{getLibFuncNameList}(f); \mathcal{O}_f \longleftarrow \mathtt{getLibFuncOpType}(\mathcal{L}_f); \mathcal{I}_f \longleftarrow  \mathtt{getFuncInstrType}(f) $
% 	Get the set of library call names $\mathcal{L}_f$, library call operation types $\mathcal{O}_f$ and function instruction type $\mathcal{I}_f$ from the signature function $f$. \;
%   \ForEach{{\upshape function} t {\upshape in } $\mathcal{T}$}{
%
%   $s^t \longleftarrow \emptyset$\; 	
%   $\mathcal{L}_t \longleftarrow \mathtt{getLibFuncNameList}(t)$ \tcp*{filter 1}
%   $s^t \pluseq w_1 * \mathtt{getLibFuncNameJaccardSim}(\mathcal{L}_f,\mathcal{L}_t)$ \;
%   $\mathcal{O}_t\longleftarrow \mathtt{getLibFuncOpType}(\mathcal{L}_t)$ \tcp*{filter 2}
%   $s^t \pluseq w_2 * \mathtt{getLibFuncOpTypeJaccardSim}(\mathcal{O}_f,\mathcal{O}_t)$  \;
%   $\mathcal{I}_t \longleftarrow  \mathtt{getFuncInstrType}(t)$  \tcp*{filter 3}
%   $s^t \pluseq w_3 * \mathtt{getFuncInstrTypeJaccardSim}(\mathcal{I}_f,\mathcal{I}_t)$ \;	
%   $S[t] = s^t$ \;
%
%}
%$\mathcal{S}^s \longleftarrow \mathtt{sortCanditTargetFunc}(S)$ \tcp*{sort target functions}
%$\mathcal{T}_c \longleftarrow \mathtt{topNTargetFunc}(\mathcal{S}^s,\mathcal{N})$ \;
%\Return ${\mathcal{T}_c}$ \;
%}
% \caption{Function Filtering Algorithm}\label{algo:pre-filt}
%\end{MyAlgo}

%\subsection{Micro Execution}\label{sec:microExe}



\input{srj-experiment}
\vspace{-3mm}
\input{srj-related-work}
\vspace{-1mm}
\input{concl}
\vspace{-1mm}
\section{Acknowledgements}
This research is supported in part by the National Research Foundation, Singapore under its National Cybersecurity R$\&$D Program (Award No. NRF2014NCR-NCR001-30).


\balance
\bibliographystyle{ieeetran}
\bibliography{srj-bibliography}




% that's all folks
\end{document}


