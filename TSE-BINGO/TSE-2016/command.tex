%\renewcommand{\algorithmicrequire}{ \textbf{Input:}} %Use Input in the format of Algorithm
%\renewcommand{\algorithmicensure}{ \textbf{Output:}} %UseOutput in the format of Algorithm
%\newcommand{\beh}{behavior\xspace}
%\newcommand{\Beh}{Behavior\xspace}
%\newcommand{\behs}{behaviors\xspace}
%\newcommand{\Behs}{Actions\xspace}
%\newcommand{\aly}{analyze\xspace}
%\newcommand{\alys}{analyzes\xspace}
%\newcommand{\tool}{\textsc{AttackClone}\xspace}
%\newcommand{\graph}{Attack Graph\xspace}
%\def\WithComments{}
\ifdefined \WithComments
	\newcommand{\xyx}[1]{\textcolor{red}{#1}}
    \newcommand{\wjj}[1]{\textcolor{blue}{\textbf{ToChange: #1}}}
    \newcommand{\ly}[1]{\textcolor{darkred}{\textbf{#1}}}
\else
	\newcommand{\xyx}[1]{\textcolor{black}{#1}}
    \newcommand{\wjj}[1]{}
    \newcommand{\ly}[1]{}
\fi
\newcommand{\todo}[1]{\textbf{\textcolor{cyan}{TODO: #1}}}
\newcommand{\note}[1]{\textcolor{red}{#1}}
\newcommand{\mahin}[1]{\textcolor{blue}{#1}}
%\newcommand{\comment}[1]{\textbf{\textcolor{green}{Comment: #1}}}
\newcommand{\alarm}[1]{\textbf{\textcolor{red}{Alarm: #1}}}
\newcommand{\deprecate}[1]{\textbf{\textcolor{red}{\{Deprecated: #1\}}}}
\newcommand{\question}[1]{\textbf{\textcolor{blue}{Question: #1}}}
\newcommand{\remove}[1]{}
\newcommand{\leftmapsto}{\leftarrow\!\shortmid}
\newcommand{\pluseq}{\mathrel{+}=}

%\st{overstriking}
\newcommand*\widefbox[1]{\fbox{\hspace{2em}#1\hspace{2em}}}


%\setitemize[0]{leftmargin=10pt}
\newcommand{\fonttt}[1]{\begin{ttfamily}#1\end{ttfamily}}
\setitemize[0]{leftmargin=10pt}

\newtheorem{mydef}{\textbf{Definition}}%[section]
\definecolor{pblue}{rgb}{0.13,0.13,1}
\definecolor{pgreen}{rgb}{0,0.5,0}
\definecolor{pred}{rgb}{0.9,0,0}
\definecolor{pgray}{rgb}{0.46,0.45,0.48}
\definecolor{ppurple}{rgb}{1,0.2,1}
\definecolor{pblack}{rgb}{0,0,0}
\lstset{
	basicstyle=\scriptsize\tt,
	tabsize=4,
	showstringspaces=false,
	columns=flexible,
	commentstyle=\color{pgreen},
  	keywordstyle=\color{pblue},
  	stringstyle=\color{ppurple},
	breaklines=true,
	language=Java,
    showspaces=false,
    numbers=left,                    % where to put the line-numbers; possible values are (none, left, right)
    numbersep=5pt,                   % how far the line-numbers are from the code
    numberstyle=\tiny\color{pblack} % the style that is used for the line-numbers
}

\makeatletter
\newcommand*{\da@rightarrow}{\mathchar"0\hexnumber@\symAMSa 4B }
\newcommand*{\da@leftarrow}{\mathchar"0\hexnumber@\symAMSa 4C }
\newcommand*{\xdashrightarrow}[2][]{%
  \mathrel{%
    \mathpalette{\da@xarrow{#1}{#2}{}\da@rightarrow{\,}{}}{}%
  }%
}
\newcommand{\xdashleftarrow}[2][]{%
  \mathrel{%
    \mathpalette{\da@xarrow{#1}{#2}\da@leftarrow{}{}{\,}}{}%
  }%
}
\newcommand*{\da@xarrow}[7]{%
  % #1: below
  % #2: above
  % #3: arrow left
  % #4: arrow right
  % #5: space left
  % #6: space right
  % #7: math style
  \sbox0{$\ifx#7\scriptstyle\scriptscriptstyle\else\scriptstyle\fi#5#1#6\m@th$}%
  \sbox2{$\ifx#7\scriptstyle\scriptscriptstyle\else\scriptstyle\fi#5#2#6\m@th$}%
  \sbox4{$#7\dabar@\m@th$}%
  \dimen@=\wd0 %
  \ifdim\wd2 >\dimen@
    \dimen@=\wd2 %
  \fi
  \count@=2 %
  \def\da@bars{\dabar@\dabar@}%
  \@whiledim\count@\wd4<\dimen@\do{%
    \advance\count@\@ne
    \expandafter\def\expandafter\da@bars\expandafter{%
      \da@bars
      \dabar@
    }%
  }%
  \mathrel{#3}%
  \mathrel{%
    \mathop{\da@bars}\limits
    \ifx\\#1\\%
    \else
      _{\copy0}%
    \fi
    \ifx\\#2\\%
    \else
      ^{\copy2}%
    \fi
  }%
  \mathrel{#4}%
}
\makeatother


%%%%%%%%%%%%%%%%%%COMMANDS TO REDUCE THE SPACE%%%%%%%%%%%%%%%%%%%%%
%\addtolength{\parskip}{-1mm}
%\addtolength{\floatsep}{-6mm}
%\addtolength{\textfloatsep}{-6mm}
%\addtolength{\abovecaptionskip}{-0.5mm}
%\addtolength{\belowcaptionskip}{-0.5mm}
%Source: http://tex.stackexchange.com/questions/60216/how-to-create-a-squiggle-arrow-with-some-text-on-it-in-tikz
\newcounter{sarrow}
\newcommand\xrsquigarrow[1]{%
\stepcounter{sarrow}%
\mathrel{\begin{tikzpicture}[baseline= {( $ (current bounding box.south) + (0,-0.5ex) $ )}]
\node[inner sep=.5ex] (\thesarrow) {$\scriptstyle #1$};
\path[draw,<-,decorate,
  decoration={zigzag,amplitude=0.7pt,segment length=1.2mm,pre=lineto,pre length=4pt}]
    (\thesarrow.south east) -- (\thesarrow.south west);
\end{tikzpicture}}%
}

